\section{Módulo Cola de mínima prioridad($\alpha$)}

El módulo cola de mínima prioridad consiste en una cola de prioridad de
elementos del tipo $\alpha$ cuya prioridad está determinada por un $nat$ de forma
tal que el elemento que se ingrese con el menor $nat$ será el de mayor prioridad.

\subsection{Especificación}

	\begin{tad}{\tadNombre{Cola de mínima prioridad($\alpha$)}}
	\tadIgualdadObservacional{c}{c'}{colaMinPrior($\alpha$)}{vacía?($c$) $\igobs$ vacía?($c'$) $\yluego$ \\
							($\neg$vacía?($c$) $\impluego$ (próximo($c$) $\igobs$ próximo($c'$) $\land$ \\
							\phantom{($\neg$vacía?($c$) $\impluego$ (}desencolar($c$) $\igobs$ desencolar($c'$))}

	\tadParametrosFormales{
		\tadEncabezadoInline{géneros}{$\alpha$}\\
		\tadEncabezadoInline{operaciones}{
			\tadOperacionInline{\argumento $<$ \argumento}{$\alpha$, $\alpha$}{bool} \hfill Relación de orden total estricto\footnotemark
		}
	}

	\footnotetext{\noindent Una relación es un orden total estricto cuando se cumple:
	\begin{description}
	 \item[Antirreflexividad:] $\neg$ $a < a$ para todo $a: \alpha$
	 \item[Antisimetría:] $(a < b \implies \neg\ b < a)$ para todo $a, b: \alpha$, $a \neq b$
	 \item[Transitividad:] $((a < b \land b < c) \implies a < c)$ para todo $a, b, c: \alpha$
	 \item[Totalidad:] $(a < b \lor b < a)$ para todo $a, b: \alpha$
	\end{description}
	}

	\tadGeneros{colaMinPrior($\alpha$)}
	\tadExporta{colaMinPrior($\alpha$), generadores, observadores}
	\tadUsa{\tadNombre{Bool}}

	\tadAlinearFunciones{desencolar}{$\alpha$,colaMinPrior($\alpha$)}

	\tadObservadores
	\tadOperacion{vacía?}{colaMinPrior($\alpha$)}{bool}{}
	\tadOperacion{próximo}{colaMinPrior($\alpha$)/c}{$\alpha$}{$\neg$ vacía?($c$)}
	\tadOperacion{desencolar}{colaMinPrior($\alpha$)/c}{colaMinPrior($\alpha$)}{$\neg$ vacía?($c$)}

	\tadGeneradores
	\tadOperacion{vacía}{}{colaMinPrior($\alpha$)}{}
	\tadOperacion{encolar}{$\alpha$,colaMinPrior($\alpha$)}{colaMinPrior($\alpha$)}{}

	\tadOtrasOperaciones
	\tadOperacion{tamaño}{colaMinPrior($\alpha$)}{nat}{}

	\tadAxiomas[\paratodo{colaMinPrior($\alpha$)}{c}, \paratodo{$\alpha$}{e}]
	\tadAlinearAxiomas{desencolar(encolar($e$, $c$))}{}

	\tadAxioma{vacía?(vacía)}{true}
	\tadAxioma{vacía?(encolar($e$, $c$))}{false}

	\tadAxioma{próximo(encolar($e$, $c$))}{\IF vacía?($c$) $\oluego$\ proximo($c$) $> e$ THEN $e$ ELSE próximo($c$) FI}

	\tadAxioma{desencolar(encolar($e$, $c$))}{\IF vacía?($c$) $\oluego$\ proximo($c$) $> e$ THEN $c$ ELSE encolar($e$, desencolar($c$)) FI}

	\end{tad}

\subsection{Interfaz}
	\tadParametrosFormales{
		\tadEncabezadoInline{géneros}{$\alpha$}
	}

	\textbf{se explica con}: \tadNombre{Cola de mínima prioridad(nat)}.

	\textbf{géneros}: \TipoVariable{colaMinPrior($\alpha$)}.

\subsubsection{Operaciones básicas de Cola de mínima prioridad}

	\InterfazFuncion{Vacía}{}{colaMinPrior($\alpha$)}
	[true]
	{$res$ $\igobs$ vacía}
	[O(1)]
	[Crea una cola de prioridad vacía]

	~

	\InterfazFuncion{Vacía?}{\In{c}{colaMinPrior($\alpha$)}}{bool}
	[true]
	{$res$ $\igobs$ vacía?(c)}
	[O(1)]
	[Devuelve \TipoVariable{true} si y sólo si la cola está vacía]

	~

	\InterfazFuncion{Desencolar}{\Inout{c}{colaMinPrior($\alpha$)}}{$\alpha$}
	[$\neg$vacía?($c$) $\land$ $c$ $\igobs$ $c_0$]
	{$res$ $\igobs$ proximo($c_0$) $\land$ $c$ $\igobs$ desencolar($c_0$)}
	[O(log(tamaño(c)))]
	[Quita el elemento más prioritario]
	[Se devuelve el elemento por copia]

	~

	\InterfazFuncion{Encolar}{\Inout{c}{colaMinPrior($\alpha$)}, \In{p}{nat}, \In{a}{$\alpha$}}{}
	[$c$ $\igobs$ $c_0$]
	{$c$ $\igobs$ encolar($p$,$c_0$)}
	[O(log(tamaño(c)))]
	[Agrega al elemento $\alpha$ con prioridad $p$ a la cola]
	[Se agrega el elemento por copia]

\subsection{Representación}

	\subsubsection{Representación de colaMinPrior}

		\begin{Estructura}{colaMinPrior($\alpha$)}[estr]
			\- \- \- \- donde \TipoVariable{estr} es \TipoVariable{dicc$_{avl}$(nat, nodoEncolados)}

			\- \- \- \- donde \TipoVariable{nodoEncolados} es
			\TipoVariable{tupla}($encolados$: \TipoVariable{cola($\alpha$)},
			$prioridad$: \TipoVariable{nat})
		\end{Estructura}

	\subsubsection{Invariante de Representación}

		\renewcommand{\labelenumi}{(\Roman{enumi})}

		\begin{enumerate}
			\item Todos los significados del diccionario tienen como clave
			el valor de $prioridad$
			\item Todos los significados del diccionario no pueden tener una
			cola vacía
		\end{enumerate}

	\Rep[estr][e]{
		\\($\forall n:$ nat) def?($n$, $e$) $\impluego$
		((obtener($n$, $e$).prioridad = $n$) $\land$
		$\neg$vacía?(obtener($n$, $e$).encolados))
	}\mbox{}

	\subsubsection{Función de Abstracción}

		\Abs[estr]{colaMinPrior}[e]{cmp}{
			(vacía?($cmp$) $\Leftrightarrow$ (\#claves($e$) = 0)) $\land$ \\
			\- $\neg$vacía?($cmp$) $\impluego$ \\
			\- \- ((próximo($cmp$) = próximo(mínimo($e$).encolados)) $\land$ \\
			\- \- (desencolar($cmp$) = desencolar(mínimo($e$).encolados)))
		}

\subsection{Algoritmos}
	\lstset{style=alg}

	\begin{lstlisting}[mathescape]
	'\alg{iVacía}{}{colaMinPrior($\alpha$)}'

	res $\leftarrow$ Vacio() '\ote{1}'

	'\ofi{O(1)}'
	\end{lstlisting}

	\begin{lstlisting}[mathescape]
	'\alg{iVacía?}{\In{c}{colaMinPrior($\alpha$)}}{bool}'

	res $\leftarrow$ ($\#$Claves($c$) = 0) '\ote{1}'

	'\ofi{O(1)}'
	\end{lstlisting}

	\begin{lstlisting}[mathescape]
	'\alg{iDesencolar}{\Inout{c}{colaMinPrior($\alpha$)}}{$\alpha$}'

	res $\leftarrow$ Copiar(Proximo(Minimo($c$).encolados)) '\ote{copy($\alpha$)}'
	Desencolar(Minimo($c$).encolados) '\ote{log(tamaño($c$))}'
	if EsVacia?(Minimo($c$).encolados) then '\ote{1}'
		Borrar($c$, Minimo($c$).prioridad) '\ote{log(tamaño($c$))}'
	end if

	'\ofi{O(log(tamano(c)) + O(copy(\alpha))}'
	\end{lstlisting}

	\begin{lstlisting}[mathescape]
	'\alg{iEncolar}{\Inout{c}{colaMinPrior($\alpha$)}, \In{p}{nat}, \In{a}{$\alpha$}}{}'
	if Definido?($p$) then '\ote{log(tamaño($c$))}'
		Encolar(Significado($c$, $p$).encolados, $a$) '\ote{log(tamaño($c$)) + copy($\alpha$)}'
	else
		nodoEncolados $nuevoNodoEncolados$ '\ote{1}'
		$nuevoNodoEncolados$.encolados $\leftarrow$ Vacia() '\ote{1}'
		$nuevoNodoEncolados$.prioridad $\leftarrow$ $p$ '\ote{1}'
		Encolar($nuevoNodoEncolados$.encolados, $a$) '\ote{copy($a$)}'
		Definir($c$, $p$, $nuevoNodoEncolados$) '\ote{log(tamaño($c$)) + copy($nodoEncolados$)}'
	end if

	'\ofi{O(log(tamano(c)) + O(copy(\alpha))}'
	\end{lstlisting}

