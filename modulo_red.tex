\section{Módulo Red}

\subsection{Interfaz}

\textbf{se explica con}: \tadNombre{red}.

\textbf{géneros}: \TipoVariable{red}.

  ~

  \InterfazFuncion{IniciarRed}{}{red}
  [true]
  {$res$ $\igobs$ iniciarRed}
  [$O(1)$]
  [Crea una red nueva]
  

  ~

  \InterfazFuncion{AgregarComputadora}{\Inout{r}{red}, \In{c}{compu}}{}
  [($r$ \igobs $r_0$) $\land$ (($\forall$ $c'$: compu) ($c'$ $\in$ computadoras($r$) $\Rightarrow$  ip($c$) $\neq$  ip($c'$)))  ]
  {$r$ $\igobs$ agregarComputadora($r_0$, $c$)) }
  [$O(L + n)$]
  [Agrega una computadora a la red]
  [La compu se agrega por copia]

  ~

  \InterfazFuncion{Conectar}{\Inout{r}{red}, \In{c}{compu}, \In{c'}{compu}, \In{i}{compu}, \In{i'}{compu}}{}
  [($r$ \igobs $r_0$) $\land$ ($c$ $\in$ computadoras($r$)) $\land$ ($c'$ $\in$ computadoras(r)) $\land$ (ip($c$) $\neq$ ip($c'$)) \\
   $\land$ ($\neg$conectadas?($r$, $c$, $c'$)) $\land$ ($\neg$usaInterfaz?($r$, $c$, $i$) $\land$ $\neg$usaInterfaz?($r$, $c'$, $i'$))]
  {$r$ $\igobs$ conectar($r_0$, $c$, $i$, $c'$, $i'$))}
  [$O(L) ?$]
  [Conecta dos computadoras]

  ~

  \InterfazFuncion{Computadoras}{\In{r}{red}}{conj(compu)}
  [true]
  {alias($res$ \igobs computadoras($r$))}
  [$O(1)$]
  [Devuelve las computadoras de la red]|
  [Devuelve una referancia no modificable]

  ~

  \InterfazFuncion{Conectadas?}{\In{r}{red}, \In{c}{compu}, \In{c'}{compu}}{bool}
  [($c$ $\in$ computadoras($r$)) $\land$ ($c'$ $\in$ computadoras($r$))]
  {$res$ \igobs conectadas?($r$, $c$, $c'$)}
  [$O(1)$]
  [Indica si dos computadoras de la red estan conectadas]

  ~

  \InterfazFuncion{InterfazUsada}{\In{r}{red}, \In{c}{compu},  \In{c'}{compu}}{interfaz}
  [conectadas?($r$, $c$, $c'$)]
  {$res$ \igobs interfazUsada($r$, $c$, $c'$)}
  [$O(L + n)$]
  [Devuelve la interfaz con la cual se conecta c con c']

  ~

  \InterfazFuncion{Vecinos}{\In{r}{red}, \In{c}{compu}}{conj(compu)}
  [$c$ $\in$ computadoras($r$)]
  {$res$ \igobs vecinos($r$, $c$)}
  [$O(n)$]
  [Devuelve el conjunto de computadoras conectadas con c]
  [Devuelve una copia de las computadoras conectadas a c]

  ~

  \InterfazFuncion{usaInterfaz?}{\In{r}{red}, \In{c}{compu}, \In{i}{interfaz}}{bool}
  [$c$ $\in$ computadoras($r$)]
  {$res$ \igobs usaInterfaz?($r$, $c$, $i$)}
  [$O(L + n)$]
  [Indica si la interfaz i es usada por la computadora c]

  ~

  \InterfazFuncion{CaminosMinimos}{\In{r}{red}, \In{c}{compu}, \In{c'}{compu}}{conj(lista(compu))}
  [($c$ $\in$ computadoras($r$)) $\land$ ($c'$ $\in$ computadoras(r))]
  {alias($res$ \igobs caminosMinimos($r$, $c$, $i$))}
  [$O(L)$]
  [Devuelve el conjunto de caminos minimos de c a c']
  [Devuelve una refencia no modificable]


  ~

  \InterfazFuncion{HayCamino?}{\In{r}{red}, \In{c}{compu}, \In{c'}{compu}}{bool}
  [($c$ $\in$ computadoras($r$)) $\land$ ($c'$ $\in$ computadoras(r))]
  {$res$ \igobs hayCamino?($r$, $c$, $i$)}
  [$O(L)$]
  [Indica si existe algún camino entre c y c']

  ~

  \InterfazFuncion{copiar}{\In{r}{red}}{red}
  [true]
  {$res$ \igobs $r$}
  [$O(?)$]
  [Devuelve una copia la red]

  ~

  \InterfazFuncion{• = •}{\In{r}{red}, \In{r'}{red}}{bool}
  [true]
  {$res$ \igobs ($r$ \igobs $r'$)}
  [$O(?)$]
  [Indica si r es igual a r']

  ~


\subsection{Representación}

  \subsubsection{Estructura}

    \begin{Estructura}{red}[estr]

      \begin{Tupla}[estr]
        \tupItem{compus}{conj(compu)}
        \tupItem{\\dns}{dicc$_{Trie}$(nodoRed)}
      \end{Tupla}

      ~

      \begin{Tupla}[nodoRed]
        \tupItem{pc}{puntero(compu)}
        \tupItem{\\caminos}{dicc$_{Trie}$(conj(lista(compu)))}
        \tupItem{\\conexiones}{dicc$_{Lineal}$(nat, puntero(nodoRed))}
      \end{Tupla}

      ~

    	\begin{Tupla}[compu]
    		\tupItem{ip}{string}%
    		\tupItem{interfaces}{conj(nat)}%
    	\end{Tupla}

    \end{Estructura}

\subsubsection{Invariante de Representación}
  \begin{enumerate}

  \item Todas los elementos de $compus$ deben tener IPs distintas.

  \item Para cada compu, el trie $dns$ define para la clave $<$IP de esa compu$>$ un \TipoVariable{nodoRed} cuyo $pc$ es puntero a esa compu.

  \item \TipoVariable{nodoRed}.$conexiones$ contiene como claves todas las
        interfaces usadas de la compu $c$ (que tienen que estar en $pc$.interfaces)

  \item Ningun nodo se conecta con si mismo.

  \item Ningun nodo se conecta a otro a traves de dos interfaces distintas.

  \item Para cada \TipoVariable{nodoRed} en $dns$, $caminos$ tiene como claves todas las
        IPs de las compus de la red (\TipoVariable{estr}.$compus$), y los significados corresponden a todos los caminos
        mínimos desde la compu $pc$ hacia la compu cuya IP es clave.

  \end{enumerate}

  \pagebreak

  \Rep[estr][e]{ (\\
    \\
    (($\forall c1, c2$: compu) ($c1 \neq c2$ $\land$ $c1 \in e$.compus $\land$ $c2 \in e$.compus) $\implies$ $c1$.ip $\neq$ $c2$.ip) $\land$ \\

    (($\forall c$: compu)($c \in e$.compus $\implies$ \\
    \- \- ( def?($c$.ip, $e$.dns) $\yluego$ obtener($c$.ip, $e$.dns).pc = puntero($c$) ) \\
    )) $\land$ \\

    (($\forall i$: string, $n$: nodoRed) ((def?($i$, $e$.dns) $\yluego$ $n$ = obtener($i$, $e$.dns)) $\implies$ \\
    \- \- ($\exists c$: compu) ($c \in e$.compus $\land$ ($n$.pc = puntero($c$))) \\
    )) $\land$ \\

    (($\forall i$: string, $n$: nodoRed) ((def?($i$, $e$.dns) $\yluego$ $n$ = obtener($i$, $e$.dns)) $\implies$ \\
    \- \- (($\forall t$: nat) (def?($t$, $n$.conexiones) $\implies$ ($t$ $\in$ $n$.pc$\rightarrow$interfaces))) \\
    )) $\land$ \\

    (($\forall i$: string, $n$: nodoRed) ((def?($i$, $e$.dns) $\yluego$ $n$ = obtener($i$, $e$.dns)) $\implies$ \\
    \- \- (($\forall t$: nat) (def?($t$, $n$.conexiones) $\impluego$ (obtener($t$, $n$.conexiones) $\neq$ puntero($n$))) ) \\
    )) $\land$ \\

    (($\forall i$: string, $n$: nodoRed) ((def?($i$, $e$.dns) $\yluego$ $n$ = obtener($i$, $e$.dns)) $\implies$ \\
    \- \- (($\forall t1, t2$: nat) (($t1 \neq t2$ $\land$ def?($t1$, $n$.conexiones) $\land$ def?($t2$, $n$.conexiones)) $\implies$ \\
    \- \- \- \- (obtener($t1$, $n$.conexiones) $\neq$ obtener($t2$, $n$.conexiones)) \\
    \- \- )) \\
    )) $\land$ \\

    (($\forall i1, i2$: string, $n1, n2$: nodoRed) (( \\
    \- \- (def?($i1$, $e$.dns) $\yluego$ $n1$ = obtener($i1$, $e$.dns)) $\land$ \\
    \- \- (def?($i2$, $e$.dns) $\yluego$ $n2$ = obtener($i2$, $e$.dns)) \\
    \- ) $\implies$ obtener($i2$, $n1$.caminos) = darCaminosMinimos($n1$, $n2$) \\
    )) \\


    ) \\
  }
\mbox{}

\tadAlinearFunciones{darCaminoMasCorto}{topologia, conj(pc), pc/ip, conj(pc), secu(segmento)}

\tadOperacion{vecinas}{nodoRed}{conj(nodoRed)}{}
\tadOperacion{auxVecinas}{nodoRed, dicc(nat, puntero(nodoRed))}{conj(nodoRed)}{}
\tadOperacion{secusDeLongK}{conj(secu($\alpha$)), nat}{conj(secu($\alpha$))}{}
\tadOperacion{longMenorSec}{conj(secu($\alpha$))/secus}{nat}{$\neg \emptyset?(secus)$}

\tadOperacion{darRutas}{nodoRed/nA, nodoRed/nB, conj(pc), secu(segmento)}{conj(secu(segmento))}{}
\tadOperacion{darRutasVecinas}{conj(pc)/vec, nodoRed/n, conj(pc), secu(segmento)}{conj(secu(segmento))}{}
\tadOperacion{darCaminosMinimos}{nodoRed/n1, nodoRed/n1}{conj(secu(compu))}{}

~


\tadAlinearAxiomas{darRutasEEEEEEWACHOOOOOOOOO}
\tadAxioma{vecinas($n$)}{auxVecinas($n$, $n$.conexiones)}

\tadAxioma{auxVecinas($n$, $cs$)}{
	\IF $\emptyset$?($cs$) THEN
		$\emptyset$
	ELSE
    Ag(obtener(dameUno(claves(cs)), cs), auxVecinas($n$, sinUno($cs$)))
	FI
}


\tadAxioma{secusDeLongK($secus$, $k$)}{
	\IF $\emptyset$?($secus$) THEN
		$\emptyset$
	ELSE{
		\IF long(dameUno($secus$)) = $k$ THEN
			dameUno($secus$) $\cup$ secusDeLongK(sinUno($secus$), $k$)
		ELSE
			secusDeLongK(sinUno($secus$), $k$)
		FI
	}
	FI
}

\tadAxioma{longMenorSec($secus$)}{
	\IF $\emptyset$?(sinUno($secus$)) THEN
		long(dameUno($secus$))
	ELSE
		min(long(dameUno($secus$)), \\
		longMenorSec(sinUno($secus$)))
	FI
}

~


\tadAxioma{darRutas($nA$, $nB$, $rec$, $ruta$)}{
	\IF $nB$ $\in$ vecinas($nA$) THEN
		Ag($ruta$ \circulito $nB$ , $\emptyset$)
	ELSE{
		\IF $\emptyset$?(vecinas($nA$) - $rec$) THEN
			$\emptyset$
		ELSE
			darRutas(dameUno(vecinas($nA$) - $rec$), \\ $nB$, Ag($nA$, $rec$),\\
							$ruta$ \circulito dameUno(vecinas($nA$) - $rec$)) $\cup$ \\
			darRutasVecinas(sinUno(vecinas($nA$) - $rec$), \\ $nB$, Ag($nA$, $rec$), \\
							$ruta$ \circulito dameUno(vecinas($nA$) - $rec$))
		FI
	}
	FI
}

\tadAxioma{darRutasVecinas($vecinas$, $n$, $rec$, $ruta$)}{
	\IF $\emptyset$?($vecinas$) THEN
		$\emptyset$
	ELSE
		darRutas(dameUno($vecinas$), $n$, $rec$, $ruta$) $\cup$ \\
		darRutasVecinas(sinUno($vecinas$), $n$, $rec$, $ruta$)
	FI
}

\tadAxioma{darCaminosMinimos($nA$, $nB$)}{
	secusDeLongK(darRutas($nA$, $nB$, $\emptyset$, <>), \\
	longMenorSec(darRutas($nA$, $nB$, $\emptyset$, <>))
}



  \subsubsection{Función de Abstracción}

   \Abs[estr]{red}[e]{r}{
    % m.estaciones $\igobs$ e.estaciones $\land$ \\
    % ($\forall c$, $s$: string)(($c$ $\in$ estaciones($e$) $\land$ $s$ $\in$ estaciones($e$) $\land$ $c < s$) $\Rightarrow$ \\
                              % ((def?($s$, obtener($c$, $e$)) $\igobs$ conectadas?($c$, $s$, $m$)) $\yluego$ \\
                              % (def?($s$, obtener($c$, $e$)) $\impluego$ (obtener($s$, obtener($c$, $e$)) $\igobs$ restriccion($c$, $s$, $m$)))))
   }
