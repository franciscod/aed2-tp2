\section{Módulo Diccionario String($\alpha$)}

\subsection{Interfaz}

\textbf{se explica con}: \tadNombre{Diccionario(string, $\alpha$)}.
\textbf{géneros}: \TipoVariable{diccString$(\alpha)$}.

Se representa mediante un árbol n-ario con invariante de trie

~

\InterfazFuncion{CrearDicc}{}{diccString$(\alpha)$}%
[true]
{$res$ $\igobs$ vacío()}
[$O(1)$]
[Crea un diccionario vacío.]
[]

~

\InterfazFuncion{Definido?}{\In{d}{diccString$(\alpha)$}, \In{c}{string})}{bool}
[true]
{$res$ $\igobs$ def?($d$, $c$)}
[$O(L)$]
[Devuelve true si la clave está definida en el diccionario y false en caso contrario.]
[]

~

\InterfazFuncion{Definir}{\In{d}{diccString$(\alpha)$}, \In{c}{string}, \In{s}{$\alpha$}}{}
[$ d \igobs d_0 $]
{$ d \igobs$ definir($c$, $s$, $d_0$)}
[$O(L)$ ]
[Define la clave $c$ con el significado $s$]
[Almacena una copia de $s$.]

~

\InterfazFuncion{Obtener}{\In{d}{diccString$(\alpha)$}, \In{c}{string}}{$\alpha$}
[def?($c$, $d$)]
{alias($res$ $\igobs$ obtener($c$, $d$))}
[$O(L)$]
[Devuelve el significado correspondiente a la clave $c$.]
[Devuelve el significado almacenado en el diccionario, por lo que $res$ es modificable si y sólo si $d$ lo es.]

~

\InterfazFuncion{• = •}{\Inout{d}{diccString($\alpha$)}, \Inout{d'}{diccString($\alpha$)}}{bool}
[true]
{$res$ $\igobs$ (d $\igobs$ d')}
[$O(L*n*(\alpha  \igobs \alpha'))$]
[Indica si d es igual d']
[]

~

\InterfazFuncion{Copiar}{\In{dicc}{diccString$(\alpha)$}}{diccString$(\alpha)$}
[true]
{$res$ $\igobs$ dicc}
[$O(n * L * copy(\alpha))$]
[Devuelve una copia del diccionario]
[]

