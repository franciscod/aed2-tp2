\documentclass[10pt, a4paper, spanish]{article}
\usepackage[paper=a4paper, left=1.5cm, right=1.5cm, bottom=1.5cm, top=3.5cm]{geometry}
\usepackage[spanish]{babel}
\selectlanguage{spanish}
\usepackage[utf8]{inputenc}
\usepackage[T1]{fontenc}
\usepackage{indentfirst}
\usepackage{fancyhdr}
\usepackage{latexsym}
\usepackage{lastpage}
\usepackage{aed2-symb,aed2-itef,aed2-tad,caratula}
\usepackage[colorlinks=true, linkcolor=blue]{hyperref}
\usepackage{calc}
\usepackage{ifthen}

\usepackage{xspace}
\usepackage{xargs}
% \usepackage{algorithm}% http://ctan.org/pkg/algorithms
% \usepackage{algpseudocode}% http://ctan.org/pkg/algorithmicx
\usepackage{verbatim}
% \usepackage{listings}

%Afanado
\newcommand{\moduloNombre}[1]{\textbf{#1}}

\let\NombreFuncion=\textsc
\let\TipoVariable=\texttt
\let\ModificadorArgumento=\textbf
\newcommand{\res}{$res$\xspace}
\newcommand{\tab}{\hspace*{7mm}}

\newcommandx{\TipoFuncion}[3]{%
  \NombreFuncion{#1}(#2) \ifx#3\empty\else $\to$ \res\,: \TipoVariable{#3}\fi%
}
\newcommand{\In}[2]{\ModificadorArgumento{in} \ensuremath{#1}\,: \TipoVariable{#2}\xspace}
\newcommand{\Out}[2]{\ModificadorArgumento{out} \ensuremath{#1}\,: \TipoVariable{#2}\xspace}
\newcommand{\Inout}[2]{\ModificadorArgumento{in/out} \ensuremath{#1}\,: \TipoVariable{#2}\xspace}
\newcommand{\Aplicar}[2]{\NombreFuncion{#1}(#2)}

\newlength{\IntFuncionLengthA}
\newlength{\IntFuncionLengthB}
\newlength{\IntFuncionLengthC}
%InterfazFuncion(nombre, argumentos, valor retorno, precondicion, postcondicion, complejidad, descripcion, aliasing)
\newcommandx{\InterfazFuncion}[9][4=true,6,7,8,9]{%
  \hangindent=\parindent
  \TipoFuncion{#1}{#2}{#3}\\%
  \textbf{Pre} $\equiv$ \{#4\}\\%
  \textbf{Post} $\equiv$ \{#5\}%
  \ifx#6\empty\else\\\textbf{Complejidad:} #6\fi%
  \ifx#7\empty\else\\\textbf{Descripción:} #7\fi%
  \ifx#8\empty\else\\\textbf{Aliasing:} #8\fi%
  \ifx#9\empty\else\\\textbf{Requiere:} #9\fi%
}
%FIN Afanado

\newcommand{\f}[1]{\text{#1}}
\renewcommand{\paratodo}[2]{$\forall~#2$: #1}

\sloppy

\hypersetup{%
 % Para que el PDF se abra a página completa.
 pdfstartview= {FitH \hypercalcbp{\paperheight-\topmargin-1in-\headheight}},
 pdfauthor={Grupo 12 - 1c2015 - Algoritmos y Estructuras de Datos II - DC - UBA},
 pdfsubject={TP2}
}

\parskip=5pt % 5pt es el tamaño de fuente

% Pongo en 0 la distancia extra entre ítemes.
\let\olditemize\itemize
\def\itemize{\olditemize\itemsep=0pt}

% Acomodo fancyhdr.
\pagestyle{fancy}
\thispagestyle{fancy}
\addtolength{\headheight}{1pt}
\lhead{Algoritmos y Estructuras de Datos II}
\rhead{$1^{\mathrm{er}}$ cuatrimestre de 2015}
\cfoot{\thepage /\pageref{LastPage}}
\renewcommand{\footrulewidth}{0.4pt}

% Encabezado
\lhead{Algoritmos y Estructuras de Datos II}
\rhead{Grupo 12}
% Pie de pagina
\renewcommand{\footrulewidth}{0.4pt}
\lfoot{Facultad de Ciencias Exactas y Naturales}
\rfoot{Universidad de Buenos Aires}

\begin{document}

% Datos de caratula
\materia{Algoritmos y Estructuras de Datos II}
\titulo{Trabajo Práctico II}
%\subtitulo{}
\grupo{Grupo: 12}

\integrante{Pondal, Iván}{078/14}{ivan.pondal@gmail.com}
\integrante{Paz, Maximiliano León}{251/14}{m4xileon@gmail.com}
\integrante{Mena, Manuel}{313/14}{manuelmena1993@gmail.com}
\integrante{Demartino, Francisco}{348/14}{demartino.francisco@gmail.com}

\maketitle
\newpage

%Indice
\tableofcontents

\pagebreak
\section{Módulo DCNet}

\subsection{Interfaz}

\textbf{se explica con}: \tadNombre{DCNet}.

\textbf{géneros}: \TipoVariable{dcnet}.

\subsubsection{Operaciones básicas de mapa}

\InterfazFuncion{iniciarDCNet}{\In{r}{red}}{dcnet}
[true]
{$res$ $\igobs$ iniciarDCNet()}
[$O(1)$]
[crea una DCNet nueva]
[]

\InterfazFuncion{crearPaquete}{\Inout{dcn}{dcnet}, \In{p}{paquete}}{}
[$\neg$( ($\exists$ $p'$:paquete)(	paqueteEnTransito($dcn$, $p'$) $\land$ 
									id($p$) = id($p'$) $\land$ 
									origen($p$) $\in$ computadoras(red($dcn$)) $\yluego$ 
									destino($p$) $\in$ computadoras(red($dcn$)) $\yluego$
									hayCamino?(red($dcn$), origen($p$), destino($p$))))]
{$res$ $\igobs$ crearPaquete($dcn$)}
[$O(1)$]
[crea una DCNet nueva]
[]

\subsection{Representación}

\subsubsection{Representación de dcnet}

\begin{Estructura}{dcnet}[estr]
	\begin{Tupla}[estr]
		\tupItem{topología}{red}%
		\tupItem{\\ compusDCNet}{vector(compuDCNet)}%
		\tupItem{\\ enEspera}{dicc$_{Trie}$(puntero(compuDCNet))}%
		\tupItem{\\ laQueMásEnvió}{puntero(compuDCNet)}%
	\end{Tupla}

	~

	\begin{Tupla}[compuDCNet]
		\tupItem{c}{compu}%
		\tupItem{\\ buffer}{conj(paquete)}%
		\tupItem{\\ encolados}{colaPrioridad(nat, paqueteDCNet)}%
		\tupItem{\\ paqueteAEnviar}{puntero(paqueteDCNet)}%
		\tupItem{\\ enviados}{nat}%
	\end{Tupla}

	~

	\begin{Tupla}[paqueteDCNet]
		\tupItem{p}{paquete}%
		\tupItem{it}{itConj(paquete)}%
		\tupItem{recorrido}{lista(compu)}%
	\end{Tupla}
\end{Estructura}




\end{document}
