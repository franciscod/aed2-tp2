\documentclass[10pt, a4paper, spanish]{article}
\usepackage[paper=a4paper, left=1.5cm, right=1.5cm, bottom=1.5cm, top=3.5cm]{geometry}
\usepackage[spanish]{babel}
\selectlanguage{spanish}
\usepackage[utf8]{inputenc}
\usepackage[T1]{fontenc}
\usepackage{indentfirst}
\usepackage{fancyhdr}
\usepackage{latexsym}
\usepackage{lastpage}
\usepackage{aed2-symb,aed2-itef,aed2-tad,caratula}
\usepackage[colorlinks=true, linkcolor=blue]{hyperref}
\usepackage{calc}
\usepackage{ifthen}

\usepackage{xspace}
\usepackage{xargs}
\usepackage{algorithm}% http://ctan.org/pkg/algorithms
\usepackage{algpseudocode}% http://ctan.org/pkg/algorithmicx
\usepackage{verbatim}
\usepackage{listings}

% Estilo para Algoritmos
\lstdefinestyle{alg}{tabsize=4, frame=single, escapeinside=\'\', framesep=10pt}

\newcommand{\alg}[3]{\hangindent=\parindent#1 (#2) \ifx#3\empty\else$\rightarrow$ res: #3\fi}
\newcommand\ote[1]{\hspace*{\fill}~\mbox{O(#1)}\penalty -9999 }
\newcommand\ofi[1]{\ensuremath{\textbf{Complejidad}: #1}}

% Afanado
\newcommand{\moduloNombre}[1]{\textbf{#1}}

\let\NombreFuncion=\textsc
\let\TipoVariable=\texttt
\let\ModificadorArgumento=\textbf
\newcommand{\res}{$res$\xspace}
\newcommand{\tab}{\hspace*{7mm}}

\newcommandx{\TipoFuncion}[3]{%
  \NombreFuncion{#1}(#2) \ifx#3\empty\else $\to$ \res\,: \TipoVariable{#3}\fi%
}
\newcommand{\In}[2]{\ModificadorArgumento{in} \ensuremath{#1}\,: \TipoVariable{#2}\xspace}
\newcommand{\Out}[2]{\ModificadorArgumento{out} \ensuremath{#1}\,: \TipoVariable{#2}\xspace}
\newcommand{\Inout}[2]{\ModificadorArgumento{in/out} \ensuremath{#1}\,: \TipoVariable{#2}\xspace}
\newcommand{\Aplicar}[2]{\NombreFuncion{#1}(#2)}

\newlength{\IntFuncionLengthA}
\newlength{\IntFuncionLengthB}
\newlength{\IntFuncionLengthC}
%InterfazFuncion(nombre, argumentos, valor retorno, precondicion, postcondicion, complejidad, descripcion, aliasing)
\newcommandx{\InterfazFuncion}[9][4=true,6,7,8,9]{%
  \hangindent=\parindent
  \TipoFuncion{#1}{#2}{#3}\\%
  \textbf{Pre} $\equiv$ \{#4\}\\%
  \textbf{Post} $\equiv$ \{#5\}%
  \ifx#6\empty\else\\\textbf{Complejidad:} #6\fi%
  \ifx#7\empty\else\\\textbf{Descripción:} #7\fi%
  \ifx#8\empty\else\\\textbf{Aliasing:} #8\fi%
  \ifx#9\empty\else\\\textbf{Requiere:} #9\fi%
}

\newenvironment{Interfaz}{%
  \parskip=2ex%
  \noindent\textbf{\Large Interfaz}%
  \par%
}{}

\newenvironment{Representacion}{%
  \vspace*{2ex}%
  \noindent\textbf{\Large Representación}%
  \vspace*{2ex}%
}{}

\newenvironment{Algoritmos}{%
  \vspace*{2ex}%
  \noindent\textbf{\Large Algoritmos}%
  \vspace*{2ex}%
}{}


\newcommand{\Titulo}[1]{
  \vspace*{1ex}\par\noindent\textbf{\large #1}\par
}

\newenvironmentx{Estructura}[2][2={estr}]{%
  \par\vspace*{2ex}%
  \TipoVariable{#1} \textbf{se representa con} \TipoVariable{#2}%
  \par\vspace*{1ex}%
}{%
  \par\vspace*{2ex}%
}%

\newboolean{EstructuraHayItems}
\newlength{\lenTupla}
\newenvironmentx{Tupla}[1][1={estr}]{%
    \settowidth{\lenTupla}{\hspace*{3mm}donde \TipoVariable{#1} es \TipoVariable{tupla}$($}%
    \addtolength{\lenTupla}{\parindent}%
    \hspace*{3mm}donde \TipoVariable{#1} es \TipoVariable{tupla}$($%
    \begin{minipage}[t]{\linewidth-\lenTupla}%
    \setboolean{EstructuraHayItems}{false}%
}{%
    $)$%
    \end{minipage}
}

\newcommandx{\tupItem}[3][1={\ }]{%
    %\hspace*{3mm}%
    \ifthenelse{\boolean{EstructuraHayItems}}{%
        ,#1%
    }{}%
    \emph{#2}: \TipoVariable{#3}%
    \setboolean{EstructuraHayItems}{true}%
}

\newcommandx{\RepFc}[3][1={estr},2={e}]{%
  \tadOperacion{Rep}{#1}{bool}{}%
  \tadAxioma{Rep($#2$)}{#3}%
}%

\newcommandx{\Rep}[3][1={estr},2={e}]{%
  \tadOperacion{Rep}{#1}{bool}{}%
  \tadAxioma{Rep($#2$)}{true \ssi #3}%
}%

\newcommandx{\Abs}[5][1={estr},3={e}]{%
  \tadOperacion{Abs}{#1/#3}{#2}{Rep($#3$)}%
  \settominwidth{\hangindent}{Abs($#3$) \igobs #4: #2 $\mid$ }%
  \addtolength{\hangindent}{\parindent}%
  Abs($#3$) \igobs #4: #2 $\mid$ #5%
}%

\newcommandx{\AbsFc}[4][1={estr},3={e}]{%
  \tadOperacion{Abs}{#1/#3}{#2}{Rep($#3$)}%
  \tadAxioma{Abs($#3$)}{#4}%
}%


%FIN Afanado

\newcommand{\f}[1]{\text{#1}}
\renewcommand{\paratodo}[2]{$\forall~#2$: #1}

\sloppy

\hypersetup{%
 % Para que el PDF se abra a página completa.
 pdfstartview= {FitH \hypercalcbp{\paperheight-\topmargin-1in-\headheight}},
 pdfauthor={Grupo 12 - 1c2015 - Algoritmos y Estructuras de Datos II - DC - UBA},
 pdfsubject={TP2}
}

\parskip=5pt % 5pt es el tamaño de fuente

% Pongo en 0 la distancia extra entre ítemes.
\let\olditemize\itemize
\def\itemize{\olditemize\itemsep=0pt}

% Acomodo fancyhdr.
\pagestyle{fancy}
\thispagestyle{fancy}
\addtolength{\headheight}{1pt}
\lhead{Algoritmos y Estructuras de Datos II}
\rhead{$1^{\mathrm{er}}$ cuatrimestre de 2015}
\cfoot{\thepage /\pageref{LastPage}}
\renewcommand{\footrulewidth}{0.4pt}

% Encabezado
\lhead{Algoritmos y Estructuras de Datos II}
\rhead{Grupo 12}
% Pie de pagina
\renewcommand{\footrulewidth}{0.4pt}
\lfoot{Facultad de Ciencias Exactas y Naturales}
\rfoot{Universidad de Buenos Aires}

\begin{document}

% Datos de caratula
\materia{Algoritmos y Estructuras de Datos II}
\titulo{Trabajo Práctico II}
%\subtitulo{}
\grupo{Grupo: 12}

\integrante{Pondal, Iván}{078/14}{ivan.pondal@gmail.com}
\integrante{Paz, Maximiliano León}{251/14}{m4xileon@gmail.com}
\integrante{Mena, Manuel}{313/14}{manuelmena1993@gmail.com}
\integrante{Demartino, Francisco}{348/14}{demartino.francisco@gmail.com}

\maketitle
\pagebreak

%Indice
\tableofcontents

\pagebreak
\section{Módulo DCNet}

\subsection{Interfaz}

\textbf{se explica con}: \tadNombre{DCNet}.

\textbf{géneros}: \TipoVariable{dcnet}.

\subsubsection{Operaciones básicas de mapa}

\InterfazFuncion{Crear}{}{dcnet}
[true]
{$res$ $\igobs$ vacio()}
[$O(1)$]
[crea un mapa nuevo]
[]

\subsection{Representación}

\subsubsection{Representación de dcnet}

\begin{Estructura}{dcnet}[estr]
	\begin{Tupla}[estr]
		\tupItem{topología}{red}%
		\tupItem{\\ vectorCompusDCNet}{vector(compuDCNet)}%
		\tupItem{\\ diccCompusDCNet}{dicc$_{trie}$(puntero(compuDCNet))}%
		\tupItem{\\ laQueMásEnvió}{puntero(compuDCNet)}%
	\end{Tupla}

	~

	\begin{Tupla}[compuDCNet]
		\tupItem{pc}{puntero(compu)}%
		\tupItem{\\ conjPaquetes}{conj(paquete)}%
		\tupItem{\\ diccPaquetesDCNet}{dicc$_{avl}$(nat, paqueteDCNet)}%
		\tupItem{\\ colaPaquetesDCNet}{colaPrioridad(nat, paqueteDCNet)}%
		\tupItem{\\ paqueteAEnviar}{paqueteDCNet}%
		\tupItem{enviados}{nat}%
	\end{Tupla}

	~

	\begin{Tupla}[paqueteDCNet]
		\tupItem{it}{itConj(paquete)}%
		\tupItem{recorrido}{lista(compu)}%
	\end{Tupla}

	~

	\begin{Tupla}[paquete]
		\tupItem{id}{nat}%
		\tupItem{prioridad}{nat}%
		\tupItem{origen}{compu}%
		\tupItem{destino}{compu}%
	\end{Tupla}

	~

	\begin{Tupla}[compu]
		\tupItem{ip}{string}%
		\tupItem{interfaces}{conj(nat)}%
	\end{Tupla}

\end{Estructura}

\subsubsection{Invariante de Representación}

\renewcommand{\labelenumi}{(\Roman{enumi})}

\begin{enumerate}
	\item Los elementos de vectorCompusDCNet son punteros a todas las compus de
		la topología
	\item Las claves de diccCompusDCNet son todos los hostnames de la topología
	\item Los significados de diccCompusDCNet son punteros que apuntan a las
		compuDCNet cuyo hostname equivale a su clave en vectorCompusDCNet
	\item laQueMásEnvió es un puntero a la compuDCNet en vectorCompusDCNet que
		más paquetes enviados tiene. Si no hay compus es NULL
	\item Todos los paquetes en conjPaquetes de cada compuDCNet tienen id único
	\item El paquete en conjPaquetes tiene que tener en su recorrido a la
		compuDCNet en la que se encuentra y no puede ser igual a su destino
	\item Las claves de diccPaquetesDCNet son los id de los paquetes en
		conjPaquetes 
	\item Los significados de diccPaquetesDCNet contienen un itConj que apunta al
		paquete con el id equivalente a su clave y en recorrido, un camino
		mínimo válido para el origen del paquete y la compu en la que se
		encuentra
	\item Si colaPaquetesDCNet no es vacía, su próximo es un paqueteDCNet que
		contiene un itConj apuntando a uno de los paquetes de conjPaquetes con
		mayor prioridad y un recorrido, que es un camino mínimo válido para el
		origen del paquete y la compu en la que se encuentra
\end{enumerate}


\pagebreak
\section{Módulo Red}

\subsection{Interfaz}

\textbf{se explica con}: \tadNombre{red}.

\textbf{géneros}: \TipoVariable{red}.

  ~

  \InterfazFuncion{IniciarRed}{}{red}
  [true]
  {$res$ $\igobs$ iniciarRed}
  [$O(1)$]
  [Crea una red nueva]
  []

  ~

  \InterfazFuncion{AgregarComputadora}{\Inout{r}{red}, \In{c}{compu} }{}
  [($r$ = $r_0$) $\land$ (($\forall$ $c'$: compu) ($c'$ $\in$ computadoras($r$) $\Rightarrow$  ip($c$) $\neq$  ip($c'$)))  ]
  {$r$ $\igobs$ agregarComputadora($r_0$, $c$)) }
  [$O(L + n)$]
  [Agrega un computadora a la red]
  []

  ~

  \InterfazFuncion{Conectar}{\Inout{r}{red}, \In{c}{compu}, \In{c'}{compu}, \In{i}{compu}, \In{i'}{compu}}{}
  [($r$ = $r_0$) $\land$ ($c$ $\in$ computadoras($r$)) $\land$ ($c'$ $\in$ computadoras(r)) $\land$ (ip($c$) $\neq$ ip($c'$)) \\
   $\land$ ($\neg$conectadas?($r$, $c$, $c'$)) $\land$ ($\neg$usaInterfaz?($r$, $c$, $i$) $\land$ $\neg$usaInterfaz?($r$, $c'$, $i'$))]
  {$r$ $\igobs$ conectar($r_0$, $c$, $i$, $c'$, $i'$))}
  [$O(L) ?$]
  [Conecta dos computadoras]
  []

  ~

  \InterfazFuncion{Computadoras}{\In{r}{red}}{conj(compu)}
  [true]  
  {$res$ = computadoras($r$)}
  [$O(1)$]

  ~

  \InterfazFuncion{Conectadas?}{\In{r}{red}, \In{c}{compu}, \In{c'}{compu}}{bool}
  [($c$ $\in$ computadoras($r$)) $\land$ ($c'$ $\in$ computadoras($r$))]
  {$res$ = conectadas?($r$, $c$, $c'$)}
  [$O(1)$]

  ~

  \InterfazFuncion{InterfazUsada}{\In{r}{red}, \In{c}{compu},  \In{c'}{compu}}{interfaz}
  [conectadas?($r$, $c$, $c'$)]
  {$res$ = interfazUsada($r$, $c$, $c'$)}
  [$O(?)$]

  ~

  \InterfazFuncion{Vecinos}{\In{r}{red}, \In{c}{compu}}{conj(compu)}
  [$c$ $\in$ computadoras($r$)]
  {$res$ = vecinos($r$, $c$)}
  [$O(n)$]

  ~

  \InterfazFuncion{usaInterfaz?}{\In{r}{red}, \In{c}{compu}, \In{i}{interfaz}}{bool}
  [$c$ $\in$ computadoras($r$)]
  {$res$ = usaInterfaz?($r$, $c$, $i$)}
  [$O(?)$]

  ~

  \InterfazFuncion{CaminosMinimos}{\In{r}{red}, \In{c}{compu}, \In{c'}{compu}}{conj(secu(compu))}
  [($c$ $\in$ computadoras($r$)) $\land$ ($c'$ $\in$ computadoras(r))]
  {$res$ = caminosMinimos($r$, $c$, $i$)}
  [$O(L)$]

  ~

  \InterfazFuncion{HayCamino?}{\In{r}{red}, \In{c}{compu}, \In{c'}{compu}}{bool}
  [($c$ $\in$ computadoras($r$)) $\land$ ($c'$ $\in$ computadoras(r))]
  {$res$ = hayCamino?($r$, $c$, $i$)}
  [$O(L)$]


\subsection{Representación}

  \subsubsection{Estructura}

    \begin{Estructura}{red}[estr]

      \begin{Tupla}[estr]
        \tupItem{compus}{conj(compu)}
        \tupItem{\\dns}{dicc$_{Trie}$(ip, nodoRed)}
      \end{Tupla}

      ~

      \begin{Tupla}[nodoRed]
        \tupItem{c}{puntero(compu)}
        \tupItem{\\caminos}{dicc$_{Trie}$(ip, conj(lista(compu)))}
        \tupItem{\\conexiones}{dicc$_{Lineal}$(interfaz, compu)}
      \end{Tupla}


    \end{Estructura}

\subsubsection{Invariante de Representación}
  \begin{enumerate}

  \item Todas las compus deben tener IPs distintas.

  \item Ninguna compu se conecta con si misma.

  \item Ninguna compu se conecta a otra a traves de dos interfaces distintas.

  \item El trie \TipoVariable{estr}.$dns$ apunta a un \TipoVariable{nodoRed}
        por cada elemento de $compus$.

  \item En cada \TipoVariable{nodoRed}, $c$ tiene que apuntar a un elemento de \TipoVariable{estr}.$compus$.

  \item Para cada \TipoVariable{nodoRed}, $caminos$ tiene como claves todas las
        IPs de las compus de la red, y los significados corresponden a todos los caminos
        mínimos desde la compu $c$ hacia la compu cuya IP es clave.

  \item \TipoVariable{nodoRed}.$conexiones$ contiene como claves todas las
        \TipoVariable{interfaz} usaconedas de la compu $c$ (que tienen que estar en $c$.$interfaces$)

  \end{enumerate}

  \Rep[estr][e]{

  %   ($\forall a$: string)($a \in e$.estaciones $<=>$ def?($a$, $e$.restricciones) ) $\land$ \\
  %   ($\forall c$, $s$: string)(def?($c$, $e$.restricciones) $\impluego$ \\
  %   ($\neg$def?($c$, obtener($c$, $e$.restricciones)) $\land$ \\
  %   (def?($s$, obtener($c$, $e$.restricciones)) $\impluego$ (def?($s$, $e$) $\land$ $s < c$)))) $\land$ \\
  %   ($\forall c$, $s$: string)(def?($c$, $e$.restricciones) $\impluego$ \\
  %   (def?($s$, obtener($c$, $e$.restricciones)) <=> <$c$, $s$> $\in$ $e$.sendas))}\mbox{}
  }

  \subsubsection{Función de Abstracción}

   \Abs[estr]{red}[e]{r}{
    % m.estaciones $\igobs$ e.estaciones $\land$ \\
    % ($\forall c$, $s$: string)(($c$ $\in$ estaciones($e$) $\land$ $s$ $\in$ estaciones($e$) $\land$ $c < s$) $\Rightarrow$ \\
                              % ((def?($s$, obtener($c$, $e$)) $\igobs$ conectadas?($c$, $s$, $m$)) $\yluego$ \\
                              % (def?($s$, obtener($c$, $e$)) $\impluego$ (obtener($s$, obtener($c$, $e$)) $\igobs$ restriccion($c$, $s$, $m$)))))
    }
  % }
\subsubsection{Función de Abstracción}




\subsection{Algoritmos}

\lstset{style=alg}
% La asignacion laQueMasEnvio <- NULL no se si esta bien porque si existe al menos una compu deberia devolver una, por mas que ninguna haya enviado todavia
\begin{lstlisting}[mathescape]
  '\alg{iIniciarRed}{}{red}'
    res.compus $\leftarrow$ Vacio() '\ote{1}'
    res.dns $\leftarrow$ Vacio() '\ote{1}'
  '\ofi{O(1)}'
\end{lstlisting}

\begin{lstlisting}[mathescape]
  '\alg{iAgregarComputadora}{\Inout{r}{red}, \In{c}{compu}}{}'
    AgregoCompuNuevaAlResto(r.dns,c)  '\ote{n*L}'
    AgregarRapido(r.compus, c) '\ote{1}'  
    Definir(r.dns, compu.ip, Tupla<&c,Vacio(),Vacio()>) '\ote{L}'
    InicializarConjCaminos(r,c) '\ote{n}'
  '\ofi{O(n*L)}'
\end{lstlisting}

\begin{lstlisting}[mathescape]
  '\alg{AgregoCompuNuevaAlResto}{\Inout{r}{red}, \In{c}{compu}}{}'
    itCompus:itConj(compu) $\leftarrow$ CrearIt(r.compus)   '\ote{1}'
    while HaySiguiente?(itCompus) do '\ote{1}'                                                   
      nr:nodoRed $\leftarrow$ Significado(r.dns,Siguiente(itCompus).ip) '\ote{L}'
      Definir(nr.caminos, c.ip, Vacio()) '\ote{L}'
      Avanzar(itCompus) '\ote{1}'
    end while  '\ote{n*L}'
  '\ofi{O(n*L)}'
\end{lstlisting}

\begin{lstlisting}[mathescape]
  '\alg{InicializarConjCaminos}{\Inout{r}{red}, \In{c}{compu}}{}'
    itCompus:itConj(compu) $\leftarrow$ CrearIt(r.compus) '\ote{1}'
    cams:diccTrie(ip,conj(lista(compu))) $\leftarrow$
     Significado(r.dns, c.ip).caminos   '\ote{1}'
    while HaySiguiente?(itCompus) do   '\ote{1}'                                                 
      Definir(cams, Siguiente(itCompus).ip, Vacio())  '\ote{L}'
      Avanzar(itCompus) '\ote{1}'
    end while '\ote{n}'
  '\ofi{O(n)}'
\end{lstlisting}

\begin{lstlisting}[mathescape]
  '\alg{iConectar}{\Inout{r}{red}, \In{c_0}{compu}, \In{c_1}{compu}, \In{i_0}{compu}, \In{i_1}{compu}}{}'
    nr0:nodoRed $\leftarrow$ Significado(r.dns, c0.ip) '\ote{L}'
    nr1:nodoRed $\leftarrow$ Significado(r.dns, c1.ip) '\ote{L}'
    DefinirRapido(rn0.conexiones, i0, nr1)'\ote{1}'
    DefinirRapido(nr1.conexiones, i1, nr0) '\ote{1}'
    CrearCaminosDelDNS(r) '\ote{n!*(n$^4$)}'
  '\ofi{O(n!*(n^4))}'
\end{lstlisting}

\begin{lstlisting}[mathescape]
  '\alg{CrearCaminosDelDNS}{\Inout{r}{red}}{}'
    itCompus:itConj(compu) $\leftarrow$ CrearIt(r.compus) '\ote{1}'
    while HaySiguiente?(itCompus) do  '\ote{1}'                                                  
      nr:nodoRed $\leftarrow$ Significado(d,Siguiente(itCompus).ip) '\ote{L}'
      AsignarCaminosMinimos(nr,r) '\ote{n!*(n$^3$)}'
      Avanzar(itCompus) '\ote{1}'
    end while '\ote{n!*(n$^4$)}'
  '\ofi{O(n!*(n^4))}'
\end{lstlisting}

\begin{lstlisting}[mathescape]
  '\alg{AsignarCaminosMinimos}{\Inout{nr}{nodoRed}}{}'
    itCompus:itConj(compu) $\leftarrow$ CrearIt(r.compus) '\ote{1}'
    while HaySiguiente?(itCompus) do  '\ote{1}'                                                
      Definir(nr.caminos, Siguiente(itCompus).ip '\ote{L}'
        CrearCaminosMinimos(nr, Siguiente(itCompus).ip)) '\ote{n!*(n$^2$)}'
      Avanzar(itCompus) '\ote{1}'
    end while '\ote{n!*(n$^3$}'
  '\ofi{O(n!*(n^3))}'
\end{lstlisting}

\begin{lstlisting}[mathescape]
  '\alg{CrearCaminosMinimos}{\In{desde}{nodoRed}, \In{hasta}{compu}}{conj(lista(compu))}'
    camMins:conj(lista(compu)) $\leftarrow$ Vacio() '\ote{1}'
    itCamMins:itConj(conj(lista(compu))) $\leftarrow$ CrearIt(camMins) '\ote{1}'
    pendientes:pila(nodoRed) $\leftarrow$ Vacia() '\ote{1}'
    ApilarConj(pendientes, Conexas(desde.conexiones))  '\ote{n}'
    visitados:conj(nodoRed) $\leftarrow$ Vacio() '\ote{1}'
    cam:lista(compu) $\leftarrow$ Vacia() '\ote{1}'
    AgregarAdelante(cam, *(desde.pc)) '\ote{1}'
    while $\neg$EsVacia?(pendientes) do  '\ote{1}'
      AgregarRapido(visitados,Siguiente(itNrs)) '\ote{1}'
      AgregarAdelante(cam, *(Siguiente(itNrs).pc)) '\ote{1}'
      if (Tope(pendientes).pc$\rightarrow$ip = hasta.ip) then '\ote{L}'
        if (Vacio?(camMins) $\lor$ (Longitud(cam) = Longitud(Siguiente
          (itCamMins)))) then '\ote{n}'
          AgregarRapido(camMins,cam) '\ote{1}'
        else
          if (Longitud(cam) < Longitud(Siguiente(itCamMins))) then '\ote{n}'
            camMins $\leftarrow$ Vacio() '\ote{1}'
            AgregarRapido(camMins,cam) '\ote{1}'
          endif '\ote{n}'
        endif '\ote{n}'
        Comienzo(cam) '\ote{1}'
      end if '\ote{n}'
      Desapilar(pendientes) '\ote{1}'
      nv:conj(nodoRed) $\leftarrow$ NoVisitadas(Tope(pendientes), visitados)  '\ote{n$^2$}'
      if(Vacia?(nv)) '\ote{(1}'
        Comienzo(cam)  '\ote{1}'
      else
        ApilarConj(pendientes, nv) '\ote{n}'
      end if '\ote{n}'
    end while '\ote{n! * n$^2$}'
  '\ofi{O(n!*(n^2))}'
\end{lstlisting}


\begin{lstlisting}[mathescape]
  '\alg{NoVisitadas}{\In{nr}{nodoRed}, \In{visitados}{conj(nodoRed)}}{conj(nodoRed)}'
    res $\leftarrow$  Vacio() '\ote{1}'
    itVecinos:itConj(nodoRed) $\leftarrow$ CrearIt(Conexas(nr.conexiones)) '\ote{n}'
    while HaySiguiente?(itVecinos) do  '\ote{1}'                                                   
      if($\neg$Pertenece?(visitados,Siguiente(itVecinos))) then '\ote{n}'
        AgregarRapido(res, Siguiente(itVecinos))  '\ote{1}'
      end if '\ote{n}'
      Avanzar(itVecinos) '\ote{1}'
    end while '\ote{n$^2$}'
  '\ofi{O(n^2)}'
\end{lstlisting}

\begin{lstlisting}[mathescape]
  '\alg{Conexas}{\In{conex}{dicc(interfaz,puntero(nodoRed))}}{conj(nodoRed)}'
    res $\leftarrow$ Vacio()  '\ote{1}'
    itVecinos :itDicc(interfaz, puntero(nodoRed))) $\leftarrow$ CrearIt(conex) '\ote{1}'
    while HaySiguiente?(itVecinos) do '\ote{1}'                                                    
      AgregarRapido(res, *SiguienteSignificado(itVecinos)) '\ote{1}'
      Avanzar(itVecinos) '\ote{1}'
    end while '\ote{n}'
  '\ofi{O(n)}'
\end{lstlisting}


\begin{lstlisting}[mathescape]
  '\alg{ApilarConj}{\Inout{pnr}{pila(nodoRed)}, \In{news}{conj(nodoRed)}}{}'
    it:itConj(nodoRed) $\leftarrow$ CrearIt(news) '\ote{1}'
    while HaySiguiente?(it) do  '\ote{1}'                                                   
      Apilar(pnr, Siguiente(it)) '\ote{1}'
      Avanzar(it) '\ote{1}'
    end while '\ote{n}'
  '\ofi{O(n)}'
\end{lstlisting}


\begin{lstlisting}[mathescape]
  '\alg{iComputadoras}{\In{r}{red}}{conj(compu)}'
    res  $\leftarrow$ r.compus  '\ote{1}'
  '\ofi{O(1)}'
\end{lstlisting}

\begin{lstlisting}[mathescape]
  '\alg{iConectadas?}{\In{r}{red}, \In{c_0}{compu}, \In{c_1}{compu}}{bool}'
    nr0:nodoRed $\leftarrow$ Significado(r.dns, c0.ip) '\ote{L}' 
    it :itDicc(interfaz, puntero(nodoRed))) $\leftarrow$ CrearIt(nr0.conexiones) '\ote{1}' 
    res $\leftarrow$ false  '\ote{1}'
    while HaySiguiente?(it) do  '\ote{1}'
      if c1.ip = SiguienteSignificado(it)->pc->ip then  '\ote{1}'
        res $\leftarrow$ true   '\ote{1}'
      end if '\ote{1}'
      Avanzar(it) '\ote{1}'
    end while '\ote{n}'
  '\ofi{O(L + n)}'
\end{lstlisting}

\begin{lstlisting}[mathescape]
  '\alg{iInterfazUsada}{\In{r}{red}, \In{c_0}{compu},  \In{c_1}{compu}}{interfaz}'
    nr0:nodoRed $\leftarrow$ Significado(r.dns, c0.ip)  '\ote{L}' 
    it :itDicc(itDicc(interfaz, puntero(nodoRed)) 
      $\leftarrow$ CrearIt(nr0.conexiones)  '\ote{1}' 
    while HaySiguiente?(it) do  '\ote{1}'  
      if c1.ip = SiguienteSignificado(it)->pc->ip then '\ote{1}'
        res $\leftarrow$ SiguienteClave(it)   '\ote{1}'
      end if '\ote{1}'
      Avanzar(it)   '\ote{1}'
    end while '\ote{n}'
  '\ofi{O(L + n)}'
\end{lstlisting}

\begin{lstlisting}[mathescape]
  '\alg{iVecinos}{\In{r}{red}, \In{c}{compu}}{conj(compu)}'
    nr:nodoRed $\leftarrow$ Significado(r.dns, c.ip)  '\ote{L}'
    res:conj(compu) $\leftarrow$ Vacio()  '\ote{1}'
    it :itDicc(itDicc(interfaz, puntero(nodoRed))  
      $\leftarrow$ CrearIt(nr.conexiones)  '\ote{1}'      
    while HaySiguiente?(it) do  '\ote{1}'
      AgregarRapido(res,*(SiguienteSignificado(it)->pc))  '\ote{1}'  
      Avanzar(it)  '\ote{1}'
    end while '\ote{n}'  
  '\ofi{O(L + n)}'
\end{lstlisting}

\begin{lstlisting}[mathescape]
  '\alg{iUsaInterfaz?}{\In{r}{red}, \In{c}{compu}, \In{i}{interfaz}}{bool}'
    nr:nodoRed $\leftarrow$ Significado(r.dns, c.ip)   '\ote{L}'
    res $\leftarrow$ Definido?(pnr.conexiones,i)  '\ote{n}'
  '\ofi{O(L + n)}'
\end{lstlisting}

\begin{lstlisting}[mathescape]
  '\alg{iCaminosMinimos}{\In{r}{red}, \In{c_0}{compu}, \In{c_1}{compu}}{conj(secu(compu))}'
    nr:nodoRed $\leftarrow$ Significado(r.dns, c0.ip) '\ote{L}'
    res $\leftarrow$  Significado(pnr.caminos, c1.ip) '\ote{L}'
  '\ofi{O(L)}'
\end{lstlisting}

\begin{lstlisting}[mathescape]
  '\alg{HayCamino?}{\In{r}{red}, \In{c_0}{compu}, \In{c_1}{compu}}{bool}'
    nr:nodoRed $\leftarrow$ Significado(r.dns, c0.ip)  '\ote{L}'
    res $\leftarrow$ $\neg$EsVacio?(Significado(pnr.caminos, c1.ip))  '\ote{L}'                                       
  '\ofi{O(L)}'
\end{lstlisting}

\begin{lstlisting}[mathescape]
  '\alg{Copiar}{\In{r}{red}}{red}'
    res:red $\leftarrow$ iIniciarRed '\ote{1}' 
    itCompus:itConj(compu) $\leftarrow$ CrearIt(r.compus) '\ote{1}'
     while HaySiguiente?(itCompus) do '\ote{1}'                                                 
      iAgregarComputadora(res,Siguiente(itCompus))  '\ote{n*L}'
      iConectar(res,Siguiente(itCompus)) '\ote{n!*(n$^4$)}'
    end while  '\ote{n!*(n$^5$)}'                              
  '\ofi{O(n!*(n^5))}'
\end{lstlisting}

\begin{lstlisting}[mathescape]
  '\alg{• = •}{\In{r_0}{red}, \In{r_1}{red}}{bool}'
    res $\leftarrow$ (r0.compus = r1.compus) $\land$ (r0.dns = r1.dns) '\ote{n$^2$}'                          
  '\ofi{O(n^2)}'
\end{lstlisting}


 


\pagebreak
\section{Módulo Cola de mínima prioridad($\alpha$)}

El módulo cola de mínima prioridad consiste en una cola de prioridad de
elementos del tipo $\alpha$ cuya prioridad está determinada por un $nat$ de forma
tal que el elemento que se ingrese con el menor $nat$ será el de mayor prioridad.

\subsection{Especificación}

	\begin{tad}{\tadNombre{Cola de mínima prioridad($\alpha$)}}
	\tadIgualdadObservacional{c}{c'}{colaMinPrior($\alpha$)}{vacía?($c$) $\igobs$ vacía?($c'$) $\yluego$ \\
							($\neg$vacía?($c$) $\impluego$ (próximo($c$) $\igobs$ próximo($c'$) $\land$ \\
							\phantom{($\neg$vacía?($c$) $\impluego$ (}desencolar($c$) $\igobs$ desencolar($c'$))}

	\tadParametrosFormales{
		\tadEncabezadoInline{géneros}{$\alpha$}\\
		\tadEncabezadoInline{operaciones}{
			\tadOperacionInline{\argumento $<$ \argumento}{$\alpha$, $\alpha$}{bool} \hfill Relación de orden total estricto\footnotemark
		}
	}

	\footnotetext{\noindent Una relación es un orden total estricto cuando se cumple:
	\begin{description}
	 \item[Antirreflexividad:] $\neg$ $a < a$ para todo $a: \alpha$
	 \item[Antisimetría:] $(a < b \implies \neg\ b < a)$ para todo $a, b: \alpha$, $a \neq b$
	 \item[Transitividad:] $((a < b \land b < c) \implies a < c)$ para todo $a, b, c: \alpha$
	 \item[Totalidad:] $(a < b \lor b < a)$ para todo $a, b: \alpha$
	\end{description}
	}

	\tadGeneros{colaMinPrior($\alpha$)}
	\tadExporta{colaMinPrior($\alpha$), generadores, observadores}
	\tadUsa{\tadNombre{Bool}}

	\tadAlinearFunciones{desencolar}{$\alpha$,colaMinPrior($\alpha$)}

	\tadObservadores
	\tadOperacion{vacía?}{colaMinPrior($\alpha$)}{bool}{}
	\tadOperacion{próximo}{colaMinPrior($\alpha$)/c}{$\alpha$}{$\neg$ vacía?($c$)}
	\tadOperacion{desencolar}{colaMinPrior($\alpha$)/c}{colaMinPrior($\alpha$)}{$\neg$ vacía?($c$)}

	\tadGeneradores
	\tadOperacion{vacía}{}{colaMinPrior($\alpha$)}{}
	\tadOperacion{encolar}{$\alpha$,colaMinPrior($\alpha$)}{colaMinPrior($\alpha$)}{}

	\tadOtrasOperaciones
	\tadOperacion{tamaño}{colaMinPrior($\alpha$)}{nat}{}

	\tadAxiomas[\paratodo{colaMinPrior($\alpha$)}{c}, \paratodo{$\alpha$}{e}]
	\tadAlinearAxiomas{desencolar(encolar($e$, $c$))}{}

	\tadAxioma{vacía?(vacía)}{true}
	\tadAxioma{vacía?(encolar($e$, $c$))}{false}

	\tadAxioma{próximo(encolar($e$, $c$))}{\IF vacía?($c$) $\oluego$\ proximo($c$) $> e$ THEN $e$ ELSE próximo($c$) FI}

	\tadAxioma{desencolar(encolar($e$, $c$))}{\IF vacía?($c$) $\oluego$\ proximo($c$) $> e$ THEN $c$ ELSE encolar($e$, desencolar($c$)) FI}

	\end{tad}

\subsection{Interfaz}
	\tadParametrosFormales{
		\tadEncabezadoInline{géneros}{$\alpha$}
	}

	\textbf{se explica con}: \tadNombre{Cola de mínima prioridad(nat)}.

	\textbf{géneros}: \TipoVariable{colaMinPrior($\alpha$)}.

\subsubsection{Operaciones básicas de Cola de mínima prioridad}

	\InterfazFuncion{Vacía}{}{colaMinPrior($\alpha$)}
	[true]
	{$res$ $\igobs$ vacía}
	[O(1)]
	[Crea una cola de prioridad vacía]

	~

	\InterfazFuncion{Vacía?}{\In{c}{colaMinPrior($\alpha$)}}{bool}
	[true]
	{$res$ $\igobs$ vacía?(c)}
	[O(1)]
	[Devuelve \TipoVariable{true} si y sólo si la cola está vacía]

	~

	\InterfazFuncion{Desencolar}{\Inout{c}{colaMinPrior($\alpha$)}}{$\alpha$}
	[$\neg$vacía?($c$) $\land$ $c$ $\igobs$ $c_0$]
	{$res$ $\igobs$ proximo($c_0$) $\land$ $c$ $\igobs$ desencolar($c_0$)}
	[O(log(tamaño(c)))]
	[Quita el elemento más prioritario]
	[Se devuelve el elemento por copia]

	~

	\InterfazFuncion{Encolar}{\Inout{c}{colaMinPrior($\alpha$)}, \In{p}{nat}, \In{a}{$\alpha$}}{}
	[$c$ $\igobs$ $c_0$]
	{$c$ $\igobs$ encolar($p$,$c_0$)}
	[O(log(tamaño(c)))]
	[Agrega al elemento $\alpha$ con prioridad $p$ a la cola]
	[Se agrega el elemento por copia]

\subsection{Representación}

	\subsubsection{Representación de colaMinPrior}

		\begin{Estructura}{colaMinPrior($\alpha$)}[estr]
			\- \- \- \- donde \TipoVariable{estr} es \TipoVariable{dicc$_{avl}$(nat, nodoEncolados)}

			\- \- \- \- donde \TipoVariable{nodoEncolados} es
			\TipoVariable{tupla}($encolados$: \TipoVariable{cola($\alpha$)},
			$prioridad$: \TipoVariable{nat})
		\end{Estructura}

	\subsubsection{Invariante de Representación}

		\renewcommand{\labelenumi}{(\Roman{enumi})}

		\begin{enumerate}
			\item Todos los significados del diccionario tienen como clave
			el valor de $prioridad$
			\item Todos los significados del diccionario no pueden tener una
			cola vacía
		\end{enumerate}

	\Rep[estr][e]{
		\\($\forall n:$ nat) def?($n$, $e$) $\impluego$
		((obtener($n$, $e$).prioridad = $n$) $\land$
		$\neg$vacía?(obtener($n$, $e$).encolados))
	}\mbox{}

	\subsubsection{Función de Abstracción}

		\Abs[estr]{colaMinPrior}[e]{cmp}{
			(vacía?($cmp$) $\Leftrightarrow$ (\#claves($e$) = 0)) $\land$ \\
			\- $\neg$vacía?($cmp$) $\impluego$ \\
			\- \- ((próximo($cmp$) = próximo(mínimo($e$).encolados)) $\land$ \\
			\- \- (desencolar($cmp$) = desencolar(mínimo($e$).encolados)))
		}

\subsection{Algoritmos}
	\lstset{style=alg}

	\begin{lstlisting}[mathescape]
	'\alg{iVacía}{}{colaMinPrior($\alpha$)}'

	res $\leftarrow$ Vacio() '\ote{1}'

	'\ofi{O(1)}'
	\end{lstlisting}

	\begin{lstlisting}[mathescape]
	'\alg{iVacía?}{\In{c}{colaMinPrior($\alpha$)}}{bool}'

	res $\leftarrow$ ($\#$Claves($c$) = 0) '\ote{1}'

	'\ofi{O(1)}'
	\end{lstlisting}

	\begin{lstlisting}[mathescape]
	'\alg{iDesencolar}{\Inout{c}{colaMinPrior($\alpha$)}}{$\alpha$}'

	res $\leftarrow$ Copiar(Proximo(Minimo($c$).encolados)) '\ote{copy($\alpha$)}'
	Desencolar(Minimo($c$).encolados) '\ote{log(tamaño($c$))}'
	if EsVacia?(Minimo($c$).encolados) then '\ote{1}'
		Borrar($c$, Minimo($c$).prioridad) '\ote{log(tamaño($c$))}'
	end if

	'\ofi{O(log(tamano(c)) + O(copy(\alpha))}'
	\end{lstlisting}

	\begin{lstlisting}[mathescape]
	'\alg{iEncolar}{\Inout{c}{colaMinPrior($\alpha$)}, \In{p}{nat}, \In{a}{$\alpha$}}{}'
	if Definido?($p$) then '\ote{log(tamaño($c$))}'
		Encolar(Significado($c$, $p$).encolados, $a$) '\ote{log(tamaño($c$)) + copy($\alpha$)}'
	else
		nodoEncolados $nuevoNodoEncolados$ '\ote{1}'
		$nuevoNodoEncolados$.encolados $\leftarrow$ Vacia() '\ote{1}'
		$nuevoNodoEncolados$.prioridad $\leftarrow$ $p$ '\ote{1}'
		Encolar($nuevoNodoEncolados$.encolados, $a$) '\ote{copy($a$)}'
		Definir($c$, $p$, $nuevoNodoEncolados$) '\ote{log(tamaño($c$)) + copy($nodoEncolados$)}'
	end if

	'\ofi{O(log(tamano(c)) + O(copy(\alpha))}'
	\end{lstlisting}


\pagebreak
\section{Módulo Diccionario AVL($\alpha$)}

\subsection{Interfaz}

\textbf{se explica con}: \tadNombre{Diccionario(nat, $\alpha$)}.

\textbf{géneros}: \TipoVariable{dicc$_{avl}$($\alpha$)}.

\subsubsection{Operaciones básicas de Diccionario AVL($\alpha$)}

	\InterfazFuncion{CrearDicc}{}{dicc$_{avl}$($\alpha$)}
	[true]
	{$res$ $\igobs$ vacío}%
	[$O(1)$]
	[Crea un diccionario vacío]
	[]

	~

	\InterfazFuncion{Definido?}{\In{c}{nat}, \In{d}{dicc$_{avl}$($\alpha$)})}{bool}
	[true]
	{$res$ $\igobs$ def?($c$, $d$)}
	[$O(log(\#claves(d)))$]
	[Devuelve \TipoVariable{true} si y sólo si la clave fue previamente definida en el diccionario]
	[]

	~

	\InterfazFuncion{Definir}{\In{c}{nat}, \In{s}{$\alpha$}, \Inout{d}{dicc$_{avl}$($\alpha$)}}{}
	[$d$ $\igobs$ $d_0$]
	{$d$ $\igobs$ definir($c$, $s$, $d_0$)}
	[$O(log(\#claves(d)) + copy(s))$]
	[Define la clave $c$ con el significado $s$ en $d$]
	[]

	~

	\InterfazFuncion{Obtener}{\In{c}{string}, \Inout{d}{dicc$_{avl}$($\alpha$)}}{$\alpha$}
	[def?($c$, $d$)]
	{alias($res$ $\igobs$ obtener($c$, $d$))}
	[$O(log(\#claves(d)))$]
	[Devuelve el significado correspondiente a la clave en el diccionario]
	[$res$ es modificable si y sólo si $d$ es modificable]

	~

	\InterfazFuncion{Mínimo}{\Inout{d}{dicc$_{avl}$($\alpha$)}}{$\alpha$}
	[\#claves($d$) > 0]
	{alias($res$ $\igobs$ obtener(claveMínima($d$), $d$))}
	[$O(log(\#claves(d)))$]
	[Devuelve el significado correspondiente a la clave de mínimo valor en el diccionario]
	[$res$ es modificable si y sólo si $d$ es modificable]

\subsubsection{Operaciones auxiliares del TAD}

\tadAlinearFunciones{darClaveMínima}{dicc(nat, $\alpha$)/d,conj(nat)/c}

	\tadOperacion{claveMínima}{dicc(nat, $\alpha$)/d}{nat}{$\#$claves(d) > 0}
	\tadOperacion{darClaveMínima}{dicc(nat, $\alpha$)/d,conj(nat)/c}{nat}{($\#$claves(d) > 0) $\land$ (c $\subseteq$ claves($d$))}

\tadAlinearAxiomas{darClaveMínima($d$, $c$)}

	\tadAxioma{claveMínima($d$)}{darClaveMínima($d$, claves($d$))}

	\tadAxioma{darClaveMínima($d$, $c$)}{
		\IF $\emptyset$?(sinUno($c$)) THEN
			dameUno($c$)
		ELSE {
			min(dameUno($c$), darClaveMínima($d$, sinUno($c$)))
			}
		FI
	}

\subsection{Representación}

	\subsubsection{Representación de dicc$_{avl}$($\alpha$)}

	\begin{Estructura}{dicc$_{avl}$($\alpha$)}[estr]
		\- \- \- \- donde \TipoVariable{estr} es \TipoVariable{puntero(nodoAvl)}

		\- \- \- \- donde \TipoVariable{nodoAvl} es
			\TipoVariable{tupla}(
				$clave$: \TipoVariable{nat},
				$data$: \TipoVariable{$\alpha$},
				$balance$: \TipoVariable{int},
				$hijos$: \TipoVariable{arreglo[2] de puntero(nodoAvl)}
			)
	\end{Estructura}


\lstset{style=alg,columns=fixed,basewidth=.5em}

	\subsubsection{Invariante de Representación}
	  \begin{enumerate}
		\item{Se mantiene el invariante de árbol binario de búsqueda para las claves de los nodos.}
		\item{Cada nodo tiene $balance$ $\in$ \{-1, 0, 1 \} donde $balance$ es:$\newline$
			* 0 si el arbol esta balanceado $\newline$
			* 1 si existe un nodo en el ultimo nivel de balance tal que tenga un hijo a la izq$\newline$
			* -1 si existe un nodo en el ultimo nivel de balance tal que tenga un hijo a la der}
		\item{Todas las claves son distintas.}

	  \end{enumerate}

	\tadAlinearFunciones{Rep}{estr}
	\tadAlinearAxiomas{Rep(e)}

	\Rep[estr][e]{ esABB($e$) $\land$ balanceadoBien($e$) $\land$ clavesDistintas($e$, vacío)}


		\tadAlinearFunciones{alturaBienwachooooo}{puntero(nodoAvl), conj(nat)}
		\tadAlinearAxiomas{clavesDistintas($n$, $cs$)}

		~

		\tadOperacion{esABB}{puntero(nodoAvl)}{bool}{}
		\tadOperacion{balanceadoBien}{puntero(nodoAvl)}{bool}{}
		\tadOperacion{clavesDistintas}{puntero(nodoAvl), conj(nat)}{bool}{}
		\tadOperacion{balanceado}{puntero(nodoAvl)}{bool}{}
		~

		\tadAxioma{esABB($n$)}{
			($n$ $\neq$ NULL) $\impluego$ ( \\
				((prim($n$$\rightarrow$hijos) $\neq$ NULL) $\impluego$ ($n$$\rightarrow$clave $>$ prim($n$.hijos) $\land$ esABB(prim($n$$\rightarrow$hijos)))) $\land$ \\
				((ult($n$$\rightarrow$hijos) $\neq$ NULL) $\impluego$ ($n$$\rightarrow$clave $<$ ult($n$$\rightarrow$hijos) $\land$  esABB(ult($n$$\rightarrow$hijos)))))
		}

		\tadAxioma{balanceadoBien($n$)}{
			(balanceado($n$) $\yluego$
			($n$ $\neq$ NULL) $\impluego$ (\\
			 \IF ((prim($n$$\rightarrow$hijos) $\neq$ NULL) $\land$ (ult($n$$\rightarrow$hijos) $\neq$ NULL)) THEN
			 	balanceadoBien(prim($n$$\rightarrow$hijos)) $\land$ balanceadoBien(ult($n$$\rightarrow$hijos)))
			 ELSE{
			 	\IF (prim($n$.hijos) $\neq$ NULL) THEN
			 		$n$$\rightarrow$balance = 1
			 	ELSE{
			 		\IF (prim($n$.hijos) $\neq$ NULL) THEN
			 			$n$$\rightarrow$balance = -1
			 		ELSE
			 			$n$$\rightarrow$balance = 0
			 		FI
			 	} FI
			} FI

		}

		\tadAxioma{clavesDistintas($n$, $cs$)}{
		($n$ $\neq$ NULL) $\impluego$ \\
				\- $n$$\rightarrow$clave $\notin$ $cs$ $\land$ \\
				clavesDistintas(prim($n$$\rightarrow$hijos), Ag($n$$\rightarrow$clave, $cs$)) $\land$ \\
				clavesDistintas(ult($n$$\rightarrow$hijos), Ag($n$$\rightarrow$clave, $cs$))

		}

		\tadAxioma{balanceado($n$)}{
		($n$ $\neq$ NULL) $\impluego$ \\
					 (\IF ((prim($n$$\rightarrow$hijos) $\neq$ NULL) $\land$ (ult($n$$\rightarrow$hijos) $\neq$ NULL)) THEN
					 	balanceado(prim($n$$\rightarrow$hijos)) $\land$ balanceado(ult($n$$\rightarrow$hijos)))
					 ELSE{
					 	\IF (prim($n$.hijos) $\neq$ NULL) THEN
					 		false
					 	ELSE{
					 		\IF (prim($n$.hijos) $\neq$ NULL) THEN
					 			false
					 		ELSE
					 			true
					 		FI
					 	} FI
					 } FI )
		}


	\subsubsection{Función de Abstracción}
	  \tadAlinearFunciones{Abs}{Estr/e}
	  \Abs[estr]{dicc(nat, $\alpha$)}[e]{d}{
		auxAbs(e, vacío)
	}

	~

	\tadAlinearFunciones{auxAbs}{puntero(nodoAvl), dicc(nat;\ $\alpha$)}
	\tadAlinearAxiomas{auxAbs($n$, $d$)}

	\tadAxioma{auxAbs($n$, $d$)}{
		\IF $n$ = NULL THEN
			$d$
		ELSE definir($n$.clave, $n$.data, auxAbs(prim($n$.hijos), auxAbs(ult($n$.hijos), $d$)))
		FI

	}

\subsection{Algoritmos}

\begin{lstlisting}[mathescape]
'\alg{insertar}{\Inout{dicc_{avl}(\alpha)}{tree}, \In{nat}{c}, \In{\alpha}{s}}{}'
	if (tree = NULL) then '\ote{1}'
		tree $\leftarrow$ crearNodo(c, s) '\ote{copy(s)}'
	else
		it: puntero(nodoAvl) $\leftarrow$ tree '\ote{1}'
		up: pila(puntero(nodoAvl)) '\ote{1}'
		upd: pila(int) '\ote{1}'

		break: bool $\leftarrow$ false '\ote{1}'
		while(break = false) '\ote{1}'
			if (it $\rightarrow$ clave < c)	then '\ote{1}'
			Apilar(upd, 1) '\ote{1}'
			else
			Apilar(upd, 0) '\ote{1}'
			end if
			Apilar(up, it) '\ote{1}'

			if (it $\rightarrow$ hijos[Tope(upd)] = NULL) '\ote{1}'
				break $\leftarrow$ true '\ote{1}'
			end if

			it $\leftarrow$ (it $\rightarrow$ hijos[Tope(upd)]) '\ote{1}'
		do

		(it $\rightarrow$ hijos[Tope(upd)]) $\leftarrow$ crearNodo(c, s) '\ote{1}'

		break $\leftarrow$ false '\ote{1}'
		while((Tamano(up) > 0) $\land$ (break = false)) '\ote{1}'
			if(Tope(upd) = 0) then '\ote{1}'
				(Tope(up) $\rightarrow$ balance) $\leftarrow$ (Tope(up)	$\rightarrow$ balance) - 1 '\ote{1}'
			else
				(Tope(up) $\rightarrow$ balance) $\leftarrow$ (Tope(up)	$\rightarrow$ balance) + 1 '\ote{1}'
			end if

			if(Tope(up) $\rightarrow$ balance = 0) then '\ote{1}'
				break $\leftarrow$ true '\ote{1}'
			else
				if(abs(Tope(up) $\rightarrow$ balance) > 1) then '\ote{1}'
					Tope(up) $\leftarrow$ insertarBalance(Tope(up), Tope(upd)) '\ote{1}'

					if(Tamano(up) > 1) then '\ote{1}'
						upTope: puntero(nodoAvl) $\leftarrow$ Tope(up) '\ote{1}'
						Desapilar(up) '\ote{1}'
						Desapilar(upd) '\ote{1}'
						(Tope(up) $\rightarrow$ hijos[Tope(upd)]) $\leftarrow$ upTope '\ote{1}'
					else
						tree $\leftarrow$ Tope(up) '\ote{1}'
					end if

					break $\leftarrow$ true '\ote{1}'
				end if
			end if
			Desapilar(up) '\ote{1}'
			Desapilar(upd) '\ote{1}'
		do '\ote{log(k)}'
end if
'\ofi{O(log(k)) + O(copy(s))}'
\end{lstlisting}

\begin{lstlisting}[mathescape]
'\alg{crearNodo}{\In{nat}{c}, \In{\alpha}{s}}{puntero(nodoAvl)}'
	hijos: arreglo_estatico[1] de puntero(nodoAvl) '\ote{1}'
	hijos[0] $\leftarrow$ NULL '\ote{1}'
	hijos[1] $\leftarrow$ NULL '\ote{1}'
	res $\leftarrow$ puntero(<c, copy(s), 0, hijos>) '\ote{copy(s)}'
	'\ofi{O(copy(s))}'
\end{lstlisting}

\begin{lstlisting}[mathescape]
'\alg{insertarBalance}{\Inout{puntero(nodoAvl)}{root}, \In{int}{dir}}{puntero(nodoAvl)}'
	nodo: puntero(nodoAvl) $\leftarrow$ (root $\rightarrow$ hijos[dir]) '\ote{1}'

	if(dir = 0) then '\ote{1}'
		bal: int $\leftarrow$ -1 '\ote{1}'
	else
		bal: int $\leftarrow$ 1 '\ote{1}'
	end if

	if(nodo $\rightarrow$ balance = bal) then '\ote{1}'
		(root $\rightarrow$ balance) $\leftarrow$ 0 '\ote{1}'
		(nodo $\rightarrow$ balance) $\leftarrow$ 0 '\ote{1}'
		root $\leftarrow$ rotacionSimple(root, $\neg$dir) '\ote{1}'
	else
		ajustarBalance(root, dir, bal) '\ote{1}'
		root $\leftarrow$ rotacionDoble(root, $\neg$dir) '\ote{1}'
	end if

	res $\leftarrow$ root '\ote{1}'
'\ofi{O(1)}'
\end{lstlisting}

\begin{lstlisting}[mathescape]
'\alg{rotacionSimple}{\Inout{puntero(nodoAvl)}{root}, \In{int}{dir}}{puntero(nodoAvl)}'
	nodo: puntero(nodoAvl) $\leftarrow$ (root $\rightarrow$ hijos[$\neg$dir]) '\ote{1}'

	(root $\rightarrow$ hijos[$\neg$dir]) $\leftarrow$ (nodo $\rightarrow$ hijos[dir]) '\ote{1}'
	(nodo $\rightarrow$ hijos[dir]) $\leftarrow$ root '\ote{1}'

	res $\leftarrow$ nodo '\ote{1}'
'\ofi{O(1)}'
\end{lstlisting}

\begin{lstlisting}[mathescape]
'\alg{rotacionDoble}{\Inout{puntero(nodoAvl)}{root}, \In{int}{dir}}{puntero(nodoAvl)}'
	nodo: puntero(nodoAvl) $\leftarrow$ ((root $\rightarrow$ hijos[$\neg$dir]) $\rightarrow$ hijos[dir]) '\ote{1}'

	((root $\rightarrow$ hijos[$\neg$dir]) $\rightarrow$ hijos[dir]) $\leftarrow$ (nodo $\rightarrow$ hijos[$\neg$dir]) '\ote{1}'
	(nodo $\rightarrow$ hijos[$\neg$dir]) $\leftarrow$ (root $\rightarrow$ hijos[$\neg$dir]) '\ote{1}'
	(root $\rightarrow$ hijos[$\neg$dir]) $\leftarrow$ nodo '\ote{1}'

	nodo $\leftarrow$ (root $\rightarrow$ hijos[$\neg$dir]) '\ote{1}'
	(root $\rightarrow$ hijos[$\neg$dir]) $\leftarrow$ (nodo $\rightarrow$ hijos[dir]) '\ote{1}'
	(nodo $\rightarrow$ hijos[dir]) $\leftarrow$ root '\ote{1}'

	res $\leftarrow$ nodo '\ote{1}'
'\ofi{O(1)}'
\end{lstlisting}

\begin{lstlisting}[mathescape]
'\alg{ajustarBalance}{\Inout{puntero(nodoAvl)}{root}, \In{int}{dir}, \In{int}{bal}}{puntero(nodoAvl)}'
	nodo: puntero(nodoAvl) $\leftarrow$ (root $\rightarrow$ hijos[dir]) '\ote{1}'
	nodoHijo: puntero(nodoAvl) $\leftarrow$ (nodoUno $\rightarrow$ hijos[$\neg$dir]) '\ote{1}'

	if(nodoHijo $\rightarrow$ balance = 0) then '\ote{1}'
		(root $\rightarrow$ balance) $\leftarrow$ 0 '\ote{1}'
		(nodo $\rightarrow$ balance) $\leftarrow$ 0 '\ote{1}'
	else
		if(nodoHijo $\rightarrow$ balance = bal) then '\ote{1}'
			(root $\rightarrow$ balance) $\leftarrow$ -bal '\ote{1}'
			(nodo $\rightarrow$ balance) $\leftarrow$ 0 '\ote{1}'
		else
			(root $\rightarrow$ balance) $\leftarrow$ 0 '\ote{1}'
			(nodo $\rightarrow$ balance) $\leftarrow$ bal '\ote{1}'
		end if
	end if

	(nodoHijo $\rightarrow$ balance)$\leftarrow$ 0 '\ote{1}'
'\ofi{O(1)}'
\end{lstlisting}

\begin{lstlisting}[mathescape]
'\alg{remover}{\Inout{dicc_{avl}(\alpha)}{tree}, \In{nat}{c}}{}'
	if(tree != NULL) then '\ote{1}'
		it: puntero(nodoAvl) $\leftarrow$ tree '\ote{1}'
		up: pila(puntero(nodoAvl)) '\ote{1}'
		upd: pila(int) '\ote{1}'

		break: bool $\leftarrow$ false '\ote{1}'
		while(break = false) '\ote{1}'
			if (it $\rightarrow$ clave = c) then '\ote{1}'
				break $\leftarrow$ true '\ote{1}'
			end if

			if (it $\rightarrow$ clave < c) then '\ote{1}'
				Apilar(upd, 1) '\ote{1}'
			else
				Apilar(upd, 0) '\ote{1}'
			end if
			Apilar(up, it) '\ote{1}'

			it $\leftarrow$ (it $\rightarrow$ hijos[Tope(upd)]) '\ote{1}'
		do '\ote{log(k)}'

		if((it $\rightarrow$ hijos[0] = NULL) $\lor$ (it $\rightarrow$ hijos[0] = NULL)) then '\ote{1}'
			if(it $\rightarrow$ hijos[0] = NULL) then '\ote{1}'
				dir: int $\leftarrow$ 1 '\ote{1}'
			else
				dir: int $\leftarrow$ 0 '\ote{1}'
			end if

			if(Tamano(up) > 1) then '\ote{1}'
				(Tope(up) $\rightarrow$ hijos[Tope(upd)]) $\leftarrow$ (it $\rightarrow$ hijos[dir]) '\ote{1}'
			else
				tree $\leftarrow$ (it $\rightarrow$ hijos[dir]) '\ote{1}'
			end if
		else
			heredero: puntero(nodoAvl) $\leftarrow$ (it $\rightarrow$ hijos[1]) '\ote{1}'

			Tope(upd) $\leftarrow$ 1 '\ote{1}'
			Tope(up) $\leftarrow$ it '\ote{1}'

			while(heredero $\rightarrow$ hijos[0] != null) '\ote{1}'
				Apilar(upd, 0) '\ote{1}'
				Apilar(up, heredero) '\ote{1}'
				heredero $\leftarrow$ (heredero $\rightarrow$ hijos[0]) '\ote{1}'
			do '\ote{log(k)}'

			(it $\rightarrow$ clave) $\leftarrow$ (heredero $\rightarrow$ clave) '\ote{1}'

			Desapilar(up) '\ote{1}'
			Desapilar(upd) '\ote{1}'

			if(Tope(up) = it) then '\ote{1}'
				(Tope(up) $\rightarrow$ hijos[1]) $\leftarrow$ (heredero $\rightarrow$ hijos[1]) '\ote{1}'
			else
				(Tope(up) $\rightarrow$ hijos[0]) $\leftarrow$ (heredero $\rightarrow$ hijos[1]) '\ote{1}'
			end if
		end if

		break $\leftarrow$ false '\ote{1}'
		while((break = false) $\land$ (Tamano(up) $\geq$ 0)) '\ote{1}'
			if(Tope(upd) != 0) then '\ote{1}'
				(Tope(up) $\rightarrow$ balance) $\leftarrow$ (Tope(up) $\rightarrow$ balance) - 1 '\ote{1}'
			else
				(Tope(up) $\rightarrow$ balance) $\leftarrow$ (Tope(up) $\rightarrow$ balance) + 1 '\ote{1}'
			end if

			if(abs(Tope(up) $\rightarrow$ balance) = 1) then '\ote{1}'
				break $\leftarrow$ true '\ote{1}'
			else
				if(abs(Tope(up) $\rightarrow$ balance) > 1) then '\ote{1}'
					Tope(up) $\leftarrow$ removerBalanceo(Tope(up), Tope(upd), \&break) '\ote{1}'
					if(Tamano(up) > 1) then '\ote{1}'
						upTope: puntero(nodoAvl) $\leftarrow$ Tope(up) '\ote{1}'
						Desapilar(up) '\ote{1}'
						Desapilar(upd) '\ote{1}'
						(Tope(up) $\rightarrow$ hijos[Tope(upd)]) $\leftarrow$ upTope '\ote{1}'
					else
						tree $\leftarrow$ Tope(up) '\ote{1}'
					end if
				end if
			end if
		do '\ote{log(k)}'
	end if
	'\ofi{O(log(k))}'
\end{lstlisting}

\begin{lstlisting}[mathescape]
'\alg{removerBalanceo}{\Inout{puntero(nodoAvl)}{root}, \In{int}{dir}, \Inout{puntero(bool)}{done}}{}'
	nodo: puntero(nodoAvl) $\leftarrow$ (root $\rightarrow$ hijos[$\neg$dir]) '\ote{1}'

	if(dir = 0) then '\ote{1}'
		bal $\leftarrow$ -1 '\ote{1}'
	else
		bal $\leftarrow$ 1 '\ote{1}'
	end if

	if(nodo $\rightarrow$ balance = -bal) then '\ote{1}'
		(root $\rightarrow$ balance) $\leftarrow$ 0 '\ote{1}'
		(nodo $\rightarrow$ balance) $\leftarrow$ 0 '\ote{1}'
		root $\leftarrow$ rotacionSimple(root, dir) '\ote{1}'
	else
		if((nodo $\rightarrow$ balance) = bal) then '\ote{1}'
			ajustarBalance(root, $\neg$dir, -bal) '\ote{1}'
			root $\leftarrow$ rotacionDoble(root, dir) '\ote{1}'
		else
			(root $\rightarrow$ balance) $\leftarrow$ -bal '\ote{1}'
			(nodo $\rightarrow$ balance) $\leftarrow$ bal '\ote{1}'
			root $\leftarrow$ rotacionSimple(root, dir) '\ote{1}'
			*done $\leftarrow$ true '\ote{1}'
		end if
	end if

	res $\leftarrow$ root '\ote{1}'
'\ofi{O(1)}'
\end{lstlisting}

\begin{lstlisting}[mathescape]
'\alg{Mínimo}{\In{dicc_{avl}(\alpha)}{d}}{$\alpha$}'

	actual:puntero(nodoAvl) $\leftarrow$ d
	hijoMenor:puntero(nodoAvl)
	done:bool $\leftarrow$ false

	while (!done) do
		hijoMenor $\leftarrow$ (actual$\rightarrow$hijos[0])

		if (hijoMenor != NULL) then
			actual $\leftarrow$ hijoMenor
		else
			res $\leftarrow$ (actual$\rightarrow$data)
			done $\leftarrow$ true
		end if
	end while
\end{lstlisting}

\begin{lstlisting}[mathescape]
'\alg{Inorder}{\In{dicc_{avl}(\alpha)}{n}}{lista(tupla(clave, significado))}'
	c:puntero(nodoAvl) $\leftarrow$ n
	p:pila(puntero(nodoAvl)) $\leftarrow$ Vacia()
	done:bool $\leftarrow$ false
	res $\leftarrow$ Vacia()

	while (!done) do
		if (c != NULL) then
			Apilar(p, c)
			c $\leftarrow$ (c$\rightarrow$hijos[0])
		else
			if !EsVacia?(p) then
				AgregarAtras(res, << Tope(p)->clave, Tope(p)->data >>)
				c $\leftarrow$ Tope(p)$\rightarrow$hijos[1]
			else
				done $\leftarrow$ true
			end if
		end if
	end while
\end{lstlisting}

\begin{lstlisting}[mathescape]
'\alg{• = •}{\In{dicc_{avl}(\alpha)}{d1}, \In{dicc_{avl}(\alpha)}{d2}}{bool}'
	res $\leftarrow$ Inorder(d1) = Inorder(d2)
\end{lstlisting}

\pagebreak
\section{Módulo Diccionario Trie($\alpha$)}

\subsection{Interfaz}

\textbf{se explica con}: \tadNombre{Diccionario(string, $\alpha$)}.
\textbf{géneros}: \TipoVariable{diccString$(\alpha)$}.

Se representa mediante un árbol n-ario con invariante de trie

~

\InterfazFuncion{CrearDicc}{}{diccString$(\alpha)$}%
[true]
{$res$ $\igobs$ vacío}
[$O(1)$]
[Crea un diccionario vacío.]
[]

~

\InterfazFuncion{Definido?}{\In{c}{string}, \In{d}{diccString$(\alpha)$})}{bool}
[true]
{$res$ $\igobs$ def?($c$, $d$)}
[$O(L)$]
[Devuelve true si la clave está definida en el diccionario y false en caso contrario.]
[]

~

\InterfazFuncion{Definir}{\In{c}{string}, \In{s}{$\alpha$}, \Inout{d}{diccString$(\alpha)$}}{}
[$ d \igobs d_0 $]
{$ d \igobs$ definir($c$, $s$, $d_0$)}
[$O(L)$ ]
[Define la clave $c$ con el significado $s$]
[Almacena una copia de $s$.]

~

\InterfazFuncion{Obtener}{\In{c}{string}, \In{d}{diccString($\alpha$)}}{$\alpha$}
[def?($c$, $d$)]
{alias($res$ $\igobs$ obtener($c$, $d$))}
[$O(L)$]
[Devuelve el significado correspondiente a la clave $c$.]
[Devuelve el significado almacenado en el diccionario, por lo que $res$ es modificable si y sólo si $d$ lo es.]

~

\InterfazFuncion{• = •}{\Inout{d}{diccString($\alpha$)}, \Inout{d'}{diccString($\alpha$)}}{bool}
[true]
{$res$ $\igobs$ (d $\igobs$ d')}
[$O(L*n*(\alpha  \igobs \alpha'))$]
[Devuelve el significado correspondiente a la clave $c$.]
[Devuelve el significado almacenado en el diccionario, por lo que $res$ es modificable si y sólo si $d$ lo es.]

~

\InterfazFuncion{Copiar}{\In{dicc}{diccString$(\alpha)$}}{diccString$(\alpha)$}
[true]
{$res$ $\igobs$ dicc}
[$O(n * L * copy(\alpha))$]
[Devuelve una copia del diccionario]
[]



\end{document}
