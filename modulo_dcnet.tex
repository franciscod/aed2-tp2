\section{Módulo DCNet}

\subsection{Interfaz}

\textbf{se explica con}: \tadNombre{DCNet}.

\textbf{géneros}: \TipoVariable{dcnet}.

\subsubsection{Operaciones básicas de DCNet}

\InterfazFuncion{iniciarDCNet}{\In{r}{red}}{dcnet}
[true]
{$res$ $\igobs$ iniciarDCNet($red$)}
[$O(1)$]
[crea una DCNet nueva]
[]
~

\InterfazFuncion{crearPaquete}{\Inout{dcn}{dcnet}, \In{p}{paquete}}{}
[$dcn$ $\igobs$ $dcn_0$ $\land$ \\
$\neg$( ($\exists$ $p'$:paquete)(	paqueteEnTransito($dcn$, $p'$) $\land$ \\
									id($p$) = id($p'$) $\land$ \\
									origen($p$) $\in$ computadoras(red($dcn$)) $\yluego$ \\
									destino($p$) $\in$ computadoras(red($dcn$)) $\yluego$ \\
									hayCamino?(red($dcn$), origen($p$), destino($p$))))]
{$dcn$ $\igobs$ crearPaquete($dcn_0$)}
[$O(L + log(k))$]
[crea un nuevo paquete]
[]
~

\InterfazFuncion{avanzarSegundo}{\Inout{dcn}{dcnet}}{}
[$dcn$ $\igobs$ $dcn_0$]
{$dcn$ $\igobs$ avanzarSegundo($dcn_0$)}
[$O(n * log(L + log(k)))$]
[avanza un segundo]
[]
~

\InterfazFuncion{red}{\In{dcn}{dcnet}}{red}
[true]
{$res$ $\igobs$ red($dcn$)}
[$O(1)$]
[devuelve la red de una DCNet]
[]
~

\InterfazFuncion{caminoRecorrido}{\In{dcn}{dcnet}, \In{p}{paquete}}{secu(compu)}
[paqueteEnTransito?($dcn$, $p$)]
{$res$ $\igobs$ caminoRecorrido($dcn$, $p$)}
[$O(n * log(n))$]
[devuelve el camino recorrido por un paquete]
[]
~

\InterfazFuncion{cantidadEnviados}{\In{dcn}{dcnet}, \In{c}{compu}}{conj(paquete)}
[c $\in$ computadoras(red($dcn$))]
{$res$ $\igobs$ cantidadEnviados($dcn$, $c$)}
[$O(1)$]
[devuelve la cantidad de paquetes enviados por una compu]
[]
~

\InterfazFuncion{enEspera}{\In{dcn}{dcnet}, \In{c}{compu}}{nat}
[c $\in$ computadoras(red($dcn$))]
{$res$ $\igobs$ enEspera($dcn$, $c$)}
[$O(L)$]
[devuelve la cola de paquetes de una compu]
[]
~

\subsection{Representación}

\subsubsection{Representación de dcnet}

\begin{Estructura}{dcnet}[estr]
	\begin{Tupla}[estr]
		\tupItem{topología}{red}%
		\tupItem{\\ compusDCNet}{vector(compuDCNet)}%
		\tupItem{\\ enEspera}{dicc$_{Trie}$(puntero(compuDCNet))}%
		\tupItem{\\ laQueMásEnvió}{puntero(compuDCNet)}%
	\end{Tupla}

	~

	\begin{Tupla}[compuDCNet]
		\tupItem{c}{compu}%
		\tupItem{\\ buffer}{conj(paquete)}%
		\tupItem{\\ encolados}{colaPrioridad(nat, paqueteDCNet)}%
		\tupItem{\\ paqueteAEnviar}{puntero(paqueteDCNet)}%
		\tupItem{\\ enviados}{nat}%
	\end{Tupla}

	~

	\begin{Tupla}[paqueteDCNet]
		\tupItem{p}{paquete}%
		\tupItem{it}{itConj(paquete)}%
		\tupItem{recorrido}{lista(compu)}%
	\end{Tupla}
\end{Estructura}


