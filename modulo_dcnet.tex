\section{Módulo DCNet}

\subsection{Interfaz}

\textbf{se explica con}: \tadNombre{DCNet}.

\textbf{géneros}: \TipoVariable{dcnet}.

\subsubsection{Operaciones básicas de mapa}

\InterfazFuncion{Crear}{}{dcnet}
[true]
{$res$ $\igobs$ vacio()}
[$O(1)$]
[crea un mapa nuevo]
[]

\subsection{Representación}

\subsubsection{Representación de dcnet}

\begin{Estructura}{dcnet}[estr]
	\begin{Tupla}[estr]
		\tupItem{topología}{red}%
		\tupItem{\\ vectorCompusDCNet}{vector(compuDCNet)}%
		\tupItem{\\ diccCompusDCNet}{dicc$_{trie}$(puntero(compuDCNet))}%
		\tupItem{\\ laQueMásEnvió}{puntero(compuDCNet)}%
	\end{Tupla}

	~

	\begin{Tupla}[compuDCNet]
		\tupItem{pc}{compu}%
		\tupItem{\\ conjPaquetes}{conj(paquete)}%
		\tupItem{\\ diccPaquetes}{dicc$_{avl}$(nat, paqueteDCNet)}%
		\tupItem{\\ colaPaquetes}{colaPrioridad(nat, paqueteDCNet)}%
		\tupItem{\\ paqueteAEnviar}{paqueteDCNet}%
		\tupItem{enviados}{nat}%
	\end{Tupla}

	~

	\begin{Tupla}[paqueteDCNet]
		\tupItem{it}{itConj(paquete)}%
		\tupItem{recorrido}{lista(compu)}%
	\end{Tupla}

	~

	\begin{Tupla}[paquete]
		\tupItem{id}{nat}%
		\tupItem{prioridad}{nat}%
		\tupItem{origen}{compu}%
		\tupItem{destino}{compu}%
	\end{Tupla}

	~

	\begin{Tupla}[compu]
		\tupItem{ip}{string}%
		\tupItem{interfaces}{conj(nat)}%
	\end{Tupla}

\end{Estructura}

\subsubsection{Invariante de Representación}

\renewcommand{\labelenumi}{(\Roman{enumi})}

\begin{enumerate}
	\item El puntero de laQueMásEnvió apuntara a la compuDCNet con mayor
		paquete salvo el caso en que no haya computadoras, donde valdrá NULL
	\item Todas las claves de enEspera son las ip de las compus en la red
	\item Los significados del diccionario enEspera apuntan a las
		compuDCNet que les corresponden en el vector compusDCNet
	\item El buffer de una compuDCNet tiene los mismo paquetes que hay en
		encolados
	\item Los paquetes paqueteDCNet que hay en encolados de un compuDCNet tienen
		como recorrido un camino mínimo de su origen a la compu del compuDCNet
	\item Los paquetes tienen id único
	\item El paquete de un paqueteDCNet debe corresponder al apuntado por itConj
\end{enumerate}

