\section{Módulo DCNet}

\subsection{Interfaz}

\textbf{se explica con}: \tadNombre{DCNet}.

\textbf{géneros}: \TipoVariable{dcnet}.

\subsubsection{Operaciones básicas de DCNet}

\InterfazFuncion{IniciarDCNet}{\In{r}{red}}{dcnet}
[true]
{$res$ $\igobs$ iniciarDCNet($red$)}
[$O(n * (n + L))$ donde n es es la cantidad de computadoras y L es la longitud de nombre de computadora mas larga]
[crea una DCNet nueva tomando una red]
[]

~

\InterfazFuncion{CrearPaquete}{\Inout{dcn}{dcnet}, \In{p}{paquete}}{}
[\\ \- $\qquad$ $dcn$ $\igobs$ $dcn_0$ $\land$
\\ \- $\qquad$$\neg$( ($\exists$ $p'$:paquete)(	paqueteEnTransito($dcn$, $p'$) $\land$ id($p$) = id($p'$) $\land$ origen($p$) $\in$ computadoras(red($dcn$)) $\yluego$
\\ \- $\qquad$ \- $\qquad$ destino($p$) $\in$ computadoras(red($dcn$)) $\yluego$ hayCamino?(red($dcn$), origen($p$), destino($p$)))) \\]
{$dcn$ $\igobs$ crearPaquete($dcn_0$)}
[$O(L + log(k))$ donde L es la longitud de nombre de computadora mas larga y k es la longitud de la cola de paquetes mas larga]
[crea un nuevo paquete]
[]

~

\InterfazFuncion{AvanzarSegundo}{\Inout{dcn}{dcnet}}{}
[$dcn$ $\igobs$ $dcn_0$]
{$dcn$ $\igobs$ avanzarSegundo($dcn_0$)}
[$O(n * (L + log(k)))$ donde n es es la cantidad de computadoras, L es la longitud de nombre de computadora mas larga y k es la longitud de la cola de paquetes mas larga]
[envia los paquetes con mayor prioridad a la siguiente compu]
[]

~

\InterfazFuncion{Red}{\In{dcn}{dcnet}}{red}
[true]
{alias($res$ $\igobs$ red($dcn$))}
[$O(1)$]
[devuelve la red de una DCNet]
[res es una referencia no modificable]

~

\InterfazFuncion{CaminoRecorrido}{\In{dcn}{dcnet}, \In{p}{paquete}}{secu(compu)}
[paqueteEnTransito?($dcn$, $p$)]
{alias($res$ $\igobs$ caminoRecorrido($dcn$, $p$))}
[$O(n * log(k))$ donde n es es la cantidad de computadoras y k es la longitud de la cola de paquetes mas larga]
[devuelve el camino recorrido por un paquete]
[res es una referencia no modificable]

~

\InterfazFuncion{CantidadEnviados}{\In{dcn}{dcnet}, \In{c}{compu}}{nat}
[c $\in$ computadoras(red($dcn$))]
{$res$ $\igobs$ cantidadEnviados($dcn$, $c$)}
[$O(L)$ donde L es la longitud de nombre de computadora mas larga]
[devuelve la cantidad de paquetes enviados por una compu]
[]

~

\InterfazFuncion{EnEspera}{\In{dcn}{dcnet}, \In{c}{compu}}{conj(paquete)}
[c $\in$ computadoras(red($dcn$))]
{alias($res$ $\igobs$ enEspera($dcn$, $c$))}
[$O(L)$ donde L es la longitud de nombre de computadora mas larga]
[devuelve el conjunto de paquetes encolados en una compu]
[res es una referencia no modificable]

~

\InterfazFuncion{PaqueteEnTransito}{\In{dcn}{dcnet}, \In{p}{paquete}}{bool}
[true]
{$res$ $\igobs$ paqueteEnTransito($dcn$, $p$)}
[$O(n * log(k))$ donde n es es la cantidad de computadoras y k es la longitud de la cola de paquetes mas larga]
[indica si el paquete está en transito]
[]

~

\InterfazFuncion{LaQueMasEnvio}{\In{dcn}{dcnet}}{compu}
[true]
{alias($res$ $\igobs$ laQueMasEnvio($dcn$))}
[$O(1)$]
[devuelve la compu que mas paquetes envió]
[res es una referencia no modificable]

~

\subsection{Representación}

\subsubsection{Representación de dcnet}

\begin{Estructura}{dcnet}[estr]
	\begin{Tupla}[estr]
		\tupItem{topología}{red}%
		\tupItem{\\ vectorCompusDCNet}{vector(compuDCNet)}%
		\tupItem{\\ diccCompusDCNet}{dicc$_{trie}$(puntero(compuDCNet))}%
		\tupItem{\\ laQueMásEnvió}{puntero(compuDCNet)}%
	\end{Tupla}

	~

	\begin{Tupla}[compuDCNet]
		\tupItem{pc}{puntero(compu)}%
		\tupItem{\\ conjPaquetes}{conj(paquete)}%
		\tupItem{\\ diccPaquetesDCNet}{dicc$_{avl}$(nat, paqueteDCNet)}%
		\tupItem{\\ colaPaquetesDCNet}{colaPrioridad(nat, paqueteDCNet)}%
		\tupItem{\\ paqueteAEnviar}{paqueteDCNet}%
		\tupItem{enviados}{nat}%
	\end{Tupla}

	~

	\begin{Tupla}[paqueteDCNet]
		\tupItem{it}{itConj(paquete)}%
		\tupItem{recorrido}{lista(compu)}%
	\end{Tupla}

	~

	\begin{Tupla}[paquete]
		\tupItem{id}{nat}%
		\tupItem{prioridad}{nat}%
		\tupItem{origen}{compu}%
		\tupItem{destino}{compu}%
	\end{Tupla}

	~

	\begin{Tupla}[compu]
		\tupItem{ip}{string}%
		\tupItem{interfaces}{conj(nat)}%
	\end{Tupla}

\end{Estructura}

\subsubsection{Invariante de Representación}

\renewcommand{\labelenumi}{(\Roman{enumi})}

\begin{enumerate}
	\item Los elementos de vectorCompusDCNet son punteros a todas las compus de
		la topología
	\item Las claves de diccCompusDCNet son todos los hostnames de la topología
	\item Los significados de diccCompusDCNet son punteros que apuntan a las
		compuDCNet cuyo hostname equivale a su clave en vectorCompusDCNet
	\item laQueMásEnvió es un puntero a la compuDCNet en vectorCompusDCNet que
		más paquetes enviados tiene. Si no hay compus es NULL
	\item Todos los paquetes en conjPaquetes de cada compuDCNet tienen id único
	\item El paquete en conjPaquetes tiene que tener en su recorrido a la
		compuDCNet en la que se encuentra y no puede ser igual a su destino
	\item Las claves de diccPaquetesDCNet son los id de los paquetes en
		conjPaquetes 
	\item Los significados de diccPaquetesDCNet contienen un itConj que apunta al
		paquete con el id equivalente a su clave y en recorrido, un camino
		mínimo válido para el origen del paquete y la compu en la que se
		encuentra
	\item Si colaPaquetesDCNet no es vacía, su próximo es un paqueteDCNet que
		contiene un itConj apuntando a uno de los paquetes de conjPaquetes con
		mayor prioridad y un recorrido, que es un camino mínimo válido para el
		origen del paquete y la compu en la que se encuentra
\end{enumerate}

