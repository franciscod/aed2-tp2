\section{Módulo DCNet}

\subsection{Interfaz}

\textbf{se explica con}: \tadNombre{DCNet}.

\textbf{géneros}: \TipoVariable{dcnet}.

\subsubsection{Operaciones básicas de DCNet}

\InterfazFuncion{IniciarDCNet}{\In{r}{red}}{dcnet}
[true]
{$res$ $\igobs$ iniciarDCNet($red$)}
[$O(n * (n + L))$ donde n es es la cantidad de computadoras y L es la longitud de nombre de computadora mas larga]
[crea una DCNet nueva tomando una red]
[]

~

\InterfazFuncion{CrearPaquete}{\Inout{dcn}{dcnet}, \In{p}{paquete}}{}
[\\ \- $\qquad$ $dcn$ $\igobs$ $dcn_0$ $\land$
\\ \- $\qquad$$\neg$( ($\exists$ $p'$:paquete)(	paqueteEnTransito($dcn$, $p'$) $\land$ id($p$) = id($p'$) $\land$ origen($p$) $\in$ computadoras(red($dcn$)) $\yluego$
\\ \- $\qquad$ \- $\qquad$ destino($p$) $\in$ computadoras(red($dcn$)) $\yluego$ hayCamino?(red($dcn$), origen($p$), destino($p$)))) \\]
{$dcn$ $\igobs$ crearPaquete($dcn_0$)}
[$O(L + log(k))$ donde L es la longitud de nombre de computadora mas larga y k es la longitud de la cola de paquetes mas larga]
[crea un nuevo paquete]
[]

~

\InterfazFuncion{AvanzarSegundo}{\Inout{dcn}{dcnet}}{}
[$dcn$ $\igobs$ $dcn_0$]
{$dcn$ $\igobs$ avanzarSegundo($dcn_0$)}
[$O(n * (L + log(k)))$ donde n es es la cantidad de computadoras, L es la longitud de nombre de computadora mas larga y k es la longitud de la cola de paquetes mas larga]
[envia los paquetes con mayor prioridad a la siguiente compu]
[]

~

\InterfazFuncion{Red}{\In{dcn}{dcnet}}{red}
[true]
{alias($res$ $\igobs$ red($dcn$))}
[$O(1)$]
[devuelve la red de una DCNet]
[res es una referencia no modificable]

~

\InterfazFuncion{CaminoRecorrido}{\In{dcn}{dcnet}, \In{p}{paquete}}{secu(compu)}
[paqueteEnTransito?($dcn$, $p$)]
{alias($res$ $\igobs$ caminoRecorrido($dcn$, $p$))}
[$O(n * log(k))$ donde n es es la cantidad de computadoras y k es la longitud de la cola de paquetes mas larga]
[devuelve el camino recorrido por un paquete]
[res es una referencia no modificable]

~

\InterfazFuncion{CantidadEnviados}{\In{dcn}{dcnet}, \In{c}{compu}}{nat}
[c $\in$ computadoras(red($dcn$))]
{$res$ $\igobs$ cantidadEnviados($dcn$, $c$)}
[$O(L)$ donde L es la longitud de nombre de computadora mas larga]
[devuelve la cantidad de paquetes enviados por una compu]
[]

~

\InterfazFuncion{EnEspera}{\In{dcn}{dcnet}, \In{c}{compu}}{conj(paquete)}
[c $\in$ computadoras(red($dcn$))]
{alias($res$ $\igobs$ enEspera($dcn$, $c$))}
[$O(L)$ donde L es la longitud de nombre de computadora mas larga]
[devuelve el conjunto de paquetes encolados en una compu]
[res es una referencia no modificable]

~

\InterfazFuncion{PaqueteEnTransito}{\In{dcn}{dcnet}, \In{p}{paquete}}{bool}
[true]
{$res$ $\igobs$ paqueteEnTransito($dcn$, $p$)}
[$O(n * log(k))$ donde n es es la cantidad de computadoras y k es la longitud de la cola de paquetes mas larga]
[indica si el paquete está en transito]
[]

~

\InterfazFuncion{LaQueMasEnvio}{\In{dcn}{dcnet}}{compu}
[true]
{alias($res$ $\igobs$ laQueMasEnvio($dcn$))}
[$O(1)$]
[devuelve la compu que mas paquetes envió]
[res es una referencia no modificable]

~

\subsection{Representación}

\subsubsection{Representación de dcnet}

\begin{Estructura}{dcnet}[estr]
	\begin{Tupla}[estr]
		\tupItem{topología}{red}%
		\tupItem{\\ vectorCompusDCNet}{vector(compuDCNet)}%
		\tupItem{\\ diccCompusDCNet}{dicc$_{trie}$(puntero(compuDCNet))}%
		\tupItem{\\ conjPaquetesDCNet}{conj(paqueteDCNet)}%
		\tupItem{\\ laQueMásEnvió}{puntero(compuDCNet)}%
	\end{Tupla}

	~

	\begin{Tupla}[compuDCNet]
		\tupItem{pc}{puntero(compu)}%
		\tupItem{\\ conjPaquetes}{conj(paquete)}%
		\tupItem{\\ diccPaquetesDCNet}{dicc$_{avl}$(nat, itConj(paqueteDCNet))}%
		\tupItem{\\ colaPaquetesDCNet}{colaPrioridad(nat, itConj(paqueteDCNet))}%
		\tupItem{\\ paqueteAEnviar}{itConj(paqueteDCNet)}%
		\tupItem{enviados}{nat}%
	\end{Tupla}

	~

	\begin{Tupla}[paqueteDCNet]
		\tupItem{it}{itConj(paquete)}%
		\tupItem{recorrido}{lista(compu)}%
	\end{Tupla}

	~

	\begin{Tupla}[paquete]
		\tupItem{id}{nat}%
		\tupItem{prioridad}{nat}%
		\tupItem{origen}{compu}%
		\tupItem{destino}{compu}%
	\end{Tupla}

	~

	\begin{Tupla}[compu]
		\tupItem{ip}{string}%
		\tupItem{interfaces}{conj(nat)}%
	\end{Tupla}

\end{Estructura}

\pagebreak

\subsubsection{Invariante de Representación}

\renewcommand{\labelenumi}{(\Roman{enumi})}

\begin{enumerate}
	\item Las compus de los elementos de vectorCompusDCNet son punteros a todas las compus de
		la topología
	\item Las claves de diccCompusDCNet son todos los hostnames de la topología
	\item Los significados de diccCompusDCNet son punteros que apuntan a las
		compuDCNet cuyo hostname equivale a su clave en vectorCompusDCNet
	\item laQueMásEnvió es un puntero a la compuDCNet en vectorCompusDCNet que
		más paquetes enviados tiene. Si no hay compus es NULL
	\item El conjPaquetesDCNet contiene tuplas con iteradores a todos los
		paquetes en tránsito en la red y sus recorridos
	\item Todos los paquetes en conjPaquetes de cada compuDCNet tienen id único
		y tanto su origen como destino existen en la topología
	\item El paquete en conjPaquetes tiene que tener en su recorrido a la
		compuDCNet en la que se encuentra y esta no puede ser igual al
		destino del recorrido
	\item Las claves de diccPaquetesDCNet son los id de los paquetes en
		conjPaquetes
	\item Los significados de diccPaquetesDCNet son un iterador al
		paqueteDCNet de conjPaquetesDCNet que contiene un iterador al
		paquete con el id equivalente a su clave y un recorrido que es uno de
		los caminos mínimos del origen del paquete a la compu en la que se
		encuentra
	\item La cantidad de enviados de una compuDCNet es igual o mayor a la
		cantidad de apariciones de esa compu en los caminos recorridos de
		paquetes en la red
\end{enumerate}

\pagebreak

\Rep[estr][e]{
	\\
	($\forall c$: compu)($c$ $\in$ computadoras($e$.topologia) $\Leftrightarrow$ \\
	\- ( \\
	\- \- ($\exists cd$: compuDCNet)
	(está?($cd$, $e$.vectorCompusDCNet) $\land$ ($cd$.pc = puntero($c$)) $\land$ \\
	\- \- ($\exists s$: string)(def?($s$, $e$.diccCompusDCNet) $\land$ ($s$ = $c$.ip))) \\
	\- ) \\
	) $\yluego$\\
	($\forall cd$: compuDCNet)(está?($cd$, $e$.vectorCompusDCNet) $\Leftrightarrow$ \\
	\- ($\exists s$: string)(($s$ = $cd$.pc$\rightarrow$ip) $\land$
		def?($s$, $e$.diccCompusDCNet) $\yluego$ \\
	\- obtener($s$, $e$.diccCompusDCNet) = puntero($cd$)) \\
	) $\yluego$\\
	($\exists cd$: compuDCNet)(está?($cd$, $e$.vectorCompusDCNet) $\yluego$ \\
	*($cd$.pc) = compuQueMásEnvió($e$.vectorCompusDCNet) $\land$ $e$.laQueMásEnvió = puntero($cd$)) $\yluego$\\
	($\forall cd_1$: compuDCNet)(está?($cd_1$, $e$.vectorCompusDCNet) $\implies$ \\
	\- ($\forall p_1$: paquete)($p_1$ $\in$ $cd_1$.conjPaquetes $\implies$ \\
	\- \- ($\forall cd_2$: compuDCNet)((está?($cd_2$, $e$.vectorCompusDCNet)
		$\land$ $cd_1$ $\neq$ $cd_2$) $\implies$ \\
	\- \- \- ($\forall p_2$: paquete)($p_2$ $\in$ $cd_2$.conjPaquetes $\implies$
		$p_1$.id $\neq$ $p_2$.id) \\
	\- \- ) \\
	\- ) \\
	) $\yluego$ \\
	($\forall cd$: compuDCNet)(está?($cd$, $e$.vectorCompusDCNet) $\implies$ \\
	\- ( \\
	\- \- ($\forall p$: paquete)($p$ $\in$ $cd$.conjPaquetes $\Leftrightarrow$ \\
	\- \- \- ( \\
	\- \- \- \- (($p$.origen $\in$ computadoras($e$.topologia) $\land$ $p$.destino
		$\in$ computadoras($e$.topologia) $\land$ \\
	\- \- \- \- $p$.destino $\neq$ *($cd$.pc)) $\yluego$ \\
	\- \- \- \- ($\exists sc$: secu(compu))($sc$ $\in$
		caminosMinimos($e$.topologia, $p$.origen, $p$.destino) $\land$
		está(*($cd$.pc), $sc$))) $\land$ \\
	\- \- \- \- ($\exists n$: nat)
		((def?($n$, $cd$.diccPaquetesDCNet) $\land$ $p$.id = $n$) $\yluego$ \\
	\- \- \- \- \- ($\exists pdn$: paqueteDCNet)($pdn$ $\in$ $e$.conjPaquetesDCNet $\land$ Siguiente($pdn$.it) = $p$ $\land$ \\
	\- \- \- \- \- \- (($p$.origen = *($cd$.pc) $\land$ $pdn$.recorrido = *($cd$.pc) $\puntito$ <>) $\lor$ \\
	\- \- \- \- \- \- ($p$.origen $\neq$ *($cd$.pc) $\land$
		$pdn$.recorrido $\in$ caminosMinimos($e$.topologia, $p$.origen,
		*($cd$.pc)))) $\land$ \\
	\- \- \- \- \- \- Siguiente(obtener($n$, $cd$.diccPaquetesDCNet)) = $pdn$ \\
	\- \- \- \- \- ) \\
	\- \- \- \- ) \\
	\- \- \- ) \\
	\- \- ) $\yluego$ \\
	\- \- ($\neg$vacía?($cd$.colaPaquetesDCNet) $\Leftrightarrow$ \\
	\- \- \- ($\exists p$: paquete)(($p$ $\in$ $cd$.conjPaquetes) $\land$
		($p$ = paqueteMásPrioridad($cd$.conjPaquetes)) $\land$ \\
	\- \- \- \- ($\exists pdn$: paqueteDCNet)(($pdn$ $\in$
		$e$.conjPaquetesDCNet) $\land$ (Siguiente($pdn$.it) = $p$) $\land$ \\
	\- \- \- \-	(Siguiente(proximo($cd$.colaPaquetesDCNet)) = $pdn$)) \\
	\- \- \- ) \\
	\- \- ) $\yluego$ \\
	\- \- ($cd$.enviados $\geq$ enviadosCompu(*($cd$.pc), $e$.vectorCompusDCNet)) \\
	\- ) \\
	)
}\mbox{}

\tadOperacion{compuQueMásEnvió}{secu(compuDCNet)/scd}{compu}{$\neg$vacía?($scd$)}
\tadOperacion{maxEnviado}{secu(compuDCNet)/scd}{nat}{$\neg$vacía?($scd$)}
\tadOperacion{enviaronK}{secu(compuDCNet),nat}{conj(compu)}{}
\tadOperacion{paqueteMásPrioridad}{conj(paquete)/cp}{paquete}{$\neg \emptyset?(cp)$}
\tadOperacion{paquetesConPrioridadK}{conj(paquete),nat}{conj(paquete)}{}
\tadOperacion{altaPrioridad}{conj(paquetes)/cp}{nat}{$\neg \emptyset?(cp)$}
\tadOperacion{enviadosCompu}{compu,secu(compuDCNet)}{nat}{}
\tadOperacion{aparicionesCompu}{compu,conj(nat)/cn,dicc(nat,itConj(paqueteDCNet))/dp}{nat}{claves($dp$) $\subseteq$ $cn$}

~

\tadAxioma{compuQueMásEnvió($scd$)}{dameUno(enviaronK($scd$, maxEnviado($scd$)))}
\tadAxioma{maxEnviado($scd$)}{
	\IF vacía?(fin($scd$)) THEN
		prim($scd$).enviados
	ELSE
		max(prim($scd$), maxEnviado(fin($scd$)))
	FI
}
\tadAxioma{enviaronK($scd$, $k$)}{
	\IF vacía?($scd$) THEN
		$\emptyset$
	ELSE {
		\IF prim($scd$).enviados = $k$ THEN
			Ag(*(prim($scd$).pc), enviaronK(fin($scd$), $k$))
		ELSE
			enviaronK(fin($scd$), $k$)
		FI
		}
	FI
}
\tadAxioma{paqueteMásPrioridad($dcn$, $cp$)}{dameUno(paquetesConPrioridadK($cp$, altaPrioridad($cp$)))}

\tadAxioma{altaPrioridad($cp$)}{
	\IF $\emptyset$?(sinUno($cp$)) THEN
		dameUno($cp$).prioridad
	ELSE
		min(dameUno($cp$).prioridad, altaPrioridad(sinUno($cp$)))
	FI
}

\tadAxioma{paquetesConPrioridadK($cp$, $k$)}{
	\IF $\emptyset$?($cp$) THEN
		$\emptyset$
	ELSE {
		\IF dameUno($cp$).prioridad = $k$ THEN
			 Ag(dameUno($cp$), paquetesConPrioridadK(sinUno($cp$), $k$))
		ELSE
			paquetesConPrioridadK(sinUno($cp$), $k$)
		FI
		}
	FI
}

\tadAxioma{enviadosCompu($c$, $scd$)}{
	\IF vacía?($scd$) THEN
		0
	ELSE {
			\IF prim($scd$) = $c$ THEN
				enviadosCompu($c$, fin($scd$))
			ELSE {
				aparicionesCompu($c$, claves(prim($scd$).diccPaquetesDCNet), \\
				prim($scd$).diccPaquetesDCNet) + enviadosCompu($c$, fin($scd$))
				}
			FI
		}
	FI
}

\tadAxioma{aparicionesCompu($c$, $cn$, $dpd$)}{
	\IF	$\emptyset$?($cn$) THEN
		0
	ELSE {
			\IF está?($c$, Siguiente(obtener(dameUno($cn$), $dpd$)).recorrido) THEN
				1 + aparicionesCompu($c$, sinUno($cn$), $dpd$)
			ELSE {
				aparicionesCompu($c$, sinUno($cn$), $dpd$)
				}
			FI
		}
	FI
}

\subsubsection{Funci\'on de Abstracci\'on}
  
\Abs[estr]{dcnet}[e]{dcn}{
red($dcn$) = $e$.topología $\land$ \\
($\forall cdn$: compuDCNet)(está?($cdn$, $e$.vectorCompusDCNet) $\impluego$ \\
\- enEspera($dcn$, *($cdn$.pc)) = $cdn$.conjPaquetes $\land$ \\
\- cantidadEnviados($dcn$, *($cdn$.pc)) = $cdn$.enviados $\land$ \\
\- ($\forall p$: paquete)($p$ $\in$ $cdn$.conjPaquetes $\impluego$ \\
\- \- caminoRecorrido($dcn$, $p$) = obtener($p$.id, $cdn$.diccPaquetesDCNet).recorrido \\ 
\- ) \\
)
}


